\section{Diskussion}
\label{sec:Diskussion}
Es treten sehr große relative Abweichungen bis über $\SI{50}{\percent}$ zwischen
den aus den Abmessungen der Körper berechneten Theoriewerten für die Trägheitsmomente
und den aus der Schwingungsdauer experimentell bestimmten Trägheitsmomenten auf.
Eine mögliche Ursache könnte in dem systematischen Fehler liegen, dass der Stab
als massenlos angenommen wurde. Desweiteren war das berechnete Eigenträgheitsmoment
größer als die theoretisch berechneten Trägheitsmomente der Körper und wurde daher
nicht in die Rechnung mit einbezogen. Dies ist sicher eine weitere Quelle aus
der Abweichungen resultieren.

Auch ist eine manuelle Zeitmessung bei einer Periodendauer von wenigen Sekunden
aufgrund der Reaktionszeit des Menschen sehr ungenau. Dies wird durch die Mittelung
über fünf Perioden nur unwesentlich verbessert.
Die Näherung der Puppe durch Zylinder ist sehr grob, um ein besseres Ergebnis
zu erhalten, müsste die Puppe durch möglichst viele Zylinder, Kegel oder Kugeln
genähert werden.

Trotz der großen Abweichungen in den Messwerten bestätigt der Versuch den
Steinerschen Satz insofern, dass das Trägheitsmoment der Puppe mit abgestreckten
Armen deutlich größer ist, als das Trägheitsmoment der Puppe mit angelegten Armen.
