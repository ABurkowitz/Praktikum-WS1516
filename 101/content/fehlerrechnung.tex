\section{Fehlerrechnung}
\label{sec:Fehlerrechnung}
Die Fehlerrechnung wird mit nachfolgenden Formeln und mit Python durchgeführt.
Der Mittelwert wird mit
\begin{equation}
  \bar{x}= \frac{1}{N}\sum_{i=1}^N x_i
  \label{eqn:mittelwert}
\end{equation}
berechnet und Standardabweichung des Mittelwerts berechnet sich nach
\begin{equation}
  \Delta\bar{x} = \sqrt{\frac{1}{n(n-1)}\sum_{i=1}^n(x_i-\bar{x_i})^2} .
  \label{eqn:standardabweichung}
\end{equation}
Der Gaußsche Fehler einer Funktion $f(x_i)$, die von den fehlerbehafteten Variablen
$x_i$ abhängt, berechnet sich nach der Formel
\begin{equation}
  \Delta f = \sqrt{\sum_{i=1}^N \left(\frac{\partial f}{\partial x_i}\right)^2
  \cdot (\Delta x_i)^2}.
  \label{eqn:fehlerfortpflanzung}
\end{equation}
Dabei ist $\Delta x_i$ der Fehler von $x_i$.
Die relative Abweichung eines berechneten Wertes $x$ vom wahren Wert $x_\symup{w}$
beträgt
\begin{equation}
  \Delta x_\symup{rel} = \frac{x-x_\symup{w}}{x_\symup{w}}.
  \label{eqn:rel_abweichung}
\end{equation}
