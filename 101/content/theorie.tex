<<<<<<< HEAD
\section{Theorie}
\label{sec:Theorie}

\cite{sample}
||||||| merged common ancestors
=======
\section{Theorie}
\label{sec:Theorie}
Mit Drehmoment $M$ Trägheitsmoment $I$ und Winkelbeschleunigung $\dot{\omega}$
lassen sich Dynamische Rotationsbewegungen beschreiben. Das Trägheitsmoment ist
durch
\begin{equation}
  I=\sum_i r_1^2m_i=\int r^2 \text{d}m
\end{equation}
gegeben.
Beispiele für trägheitsmomente sind
\begin{figure}
\end{figure}
Ist ein Körper dessen Trägheitsmoment bekannt ist, von der Drehachse um
$a$ verschoben so lasst sich das neue Trägheitsmoment berechnen durch den
Steinerschen satz
\begin{equation}
  I=I_s+m a^2
\end{equation}
>>>>>>> 354 1 von 2 fits und 103 berichtigt
