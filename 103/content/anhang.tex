\begin{table}
  \centering
  \begin{tabular}{c c c}
    \toprule
    x in \si{\centi\meter} & $D_0(x)$ in \si{\milli\meter} & $D_M(x)$ in \si{\milli\meter} \\
    \midrule
    48.0	&	-0.28	&	4.74 \\
    47.5	&	-0.26	&	4.69 \\
    47.0	&	-0.24	&	4.64 \\
    46.5	&	-0.13	&	4.58 \\
    46.0	&	-0.22	&	4.51 \\
    45.5	&	-0.20	&	4.45 \\
    45.0	&	-0.19	&	4.39 \\
    44.5	&	-0.19	&	4.32 \\
    44.0	&	-0.17	&	4.26 \\
    43.5	&	-0.14	&	4.20 \\
    43.0	&	-0.14	&	4.14 \\
    42.5	&	-0.11	&	4.08 \\
    42.0	&	-0.11	&	4.01 \\
    41.5	&	-0.09	&	3.94 \\
    41.0	&	-0.08	&	3.87 \\
    40.5	&	-0.07	&	3.81 \\
    40.0	&	-0.05	&	3.76 \\
    39.0	&	-0.01	&	3.65 \\
    38.0	&	0.04	&	3.55 \\
    37.0	&	0.08	&	3.44 \\
    36.0	&	0.14	&	3.35 \\
    35.0	&	0.18	&	3.28 \\
    34.0	&	0.23	&	3.16 \\
    33.0	&	0.28	&	3.07 \\
    32.0	&	0.32	&	2.97 \\
    31.0	&	0.38	&	2.90 \\
    30.0	&	0.44	&	2.82 \\
    28.0	&	0.55	&	2.68 \\
    26.0	&	0.69	&	2.56 \\
    24.0	&	0.82	&	2.45 \\
    22.0	&	0.96	&	2.36 \\
    20.0	&	1.10	&	2.28 \\
    15.0	&	1.43	&	2.14 \\
    10.0	&	1.84	&	2.19 \\
     5.0	&	0.60	&	0.70 \\
     2.5	&	0.85	&	0.82 \\
    \bottomrule
  \end{tabular}
  \caption{Messwerte zur Berechnung von $D(x)$ für den eckigen Stab 1 bei einseitiger
  Einspannung.}
  \label{tab:messung1}
\end{table}


\begin{table}
  \centering
  \begin{tabular}{c c c}
    \toprule
    x in \si{\centi\meter} & $D_0(x)$ in \si{\milli\meter} & $D_M(x)$ in \si{\milli\meter} \\
    \midrule
    48.0 & 1.70 & 6.25 \\
    47.5 & 1.60 & 6.17 \\
    47.0 & 1.60 & 6.09 \\
    46.5 & 1.58 & 6.01 \\
    46.0 & 1.58 & 5.92 \\
    45.5 & 1.56 & 5.82 \\
    45.0 & 1.55 & 5.73 \\
    44.5 & 1.53 & 5.65 \\
    44.0 & 1.52 & 5.57 \\
    43.5 & 1.51 & 5.48 \\
    43.0 & 1.49 & 5.40 \\
    42.5 & 1.49 & 5.33 \\
    42.0 & 1.48 & 5.26 \\
    41.5 & 1.47 & 5.17 \\
    41.0 & 1.45 & 5.09 \\
    40.5 & 1.44 & 5.01 \\
    40.0 & 1.44 & 4.94 \\
    39.0 & 1.43 & 4.77 \\
    38.0 & 1.42 & 4.64 \\
    37.0 & 1.41 & 4.50 \\
    36.0 & 1.41 & 4.38 \\
    35.0 & 1.40 & 4.24 \\
    34.0 & 1.40 & 4.11 \\
    33.0 & 1.40 & 3.98 \\
    32.0 & 1.40 & 3.85 \\
    31.0 & 1.42 & 3.75 \\
    30.0 & 1.43 & 3.63 \\
    28.0 & 1.48 & 3.42 \\
    26.0 & 1.51 & 3.25 \\
    24.0 & 1.55 & 3.06 \\
    22.0 & 1.61 & 2.91 \\
    20.0 & 1.67 & 2.75 \\
    15.0 & 1.84 & 2.50 \\
    10.0 & 2.08 & 2.39 \\
    5.0  & 0.61 & 0.69 \\
    2.5  & 0.75 & 0.78 \\
    \bottomrule
  \end{tabular}
  \caption{Messwerte zur Berechnung von $D(x)$ für den runden Stab 2 bei einseitiger
  Einspannung.}
  \label{tab:messung2}
\end{table}

\begin{table}
  \centering
  \begin{tabular}{c c c}
    \toprule
    x in \si{\centi\meter} & $D_0(x)$ in \si{\milli\meter} & $D_M(x)$ in \si{\milli\meter} \\
    \midrule
    26.6 & 1.44 & 3.52 \\
    26.1 & 1.45 & 3.52 \\
    25.6 & 1.44 & 3.50 \\
    25.1 & 1.43 & 3.49 \\
    24.6 & 1.42 & 3.47 \\
    24.1 & 1.42 & 3.46 \\
    23.6 & 1.42 & 3.44 \\
    23.1 & 1.41 & 3.41 \\
    22.6 & 1.41 & 3.39 \\
    22.1 & 1.40 & 3.37 \\
    21.1 & 1.39 & 3.31 \\
    20.1 & 1.37 & 3.24 \\
    19.1 & 1.36 & 3.27 \\
    18.1 & 1.34 & 3.09 \\
    17.1 & 1.32 & 3.00 \\
    16.1 & 1.30 & 3.10 \\
    15.1 & 1.28 & 2.80 \\
    10.1 & 1.17 & 2.27 \\
    5.1  & 1.08 & 1.66 \\
    2.5  & 1.11 & 1.42 \\
    \bottomrule
  \end{tabular}
  \caption{Messwerte zur Berechnung von $D(x)$ für den rechteckigen Stab 1 bei
  beidseitiger Einspannung und $0\leq x \leq \frac{L}{2}$.}
  \label{tab:messung3a}
\end{table}

\begin{table}
  \centering
  \begin{tabular}{c c c}
    \toprule
    x in \si{\centi\meter} & $D_0(x)$ in \si{\milli\meter} & $D_M(x)$ in \si{\milli\meter} \\
    \midrule
    28.6 & 3.15 & 5.23 \\
    29.1 & 3.15 & 5.22 \\
    29.6 & 3.15 & 5.22 \\
    30.1 & 3.17 & 5.23 \\
    30.6 & 3.18 & 5.22 \\
    31.1 & 3.18 & 5.22 \\
    31.6 & 3.18 & 5.22 \\
    32.1 & 3.19 & 5.21 \\
    32.6 & 3.20 & 5.21 \\
    33.1 & 3.21 & 5.19 \\
    34.1 & 3.22 & 5.17 \\
    35.1 & 3.25 & 5.14 \\
    36.1 & 3.26 & 5.10 \\
    37.1 & 3.29 & 5.05 \\
    38.1 & 3.29 & 5.00 \\
    39.1 & 3.30 & 4.94 \\
    40.1 & 3.31 & 4.88 \\
    45.1 & 3.37 & 4.50 \\
    50.1 & 3.36 & 3.99 \\
    52.5 & 3.36 & 3.70 \\
    \bottomrule
  \end{tabular}
  \caption{Messwerte zur Berechnung von $D(x)$ für den rechteckigen Stab 1 bei
  beidseitiger Einspannung und $\frac{L}{2}\leq x \leq L$.}
  \label{tab:messung3b}
\end{table}
