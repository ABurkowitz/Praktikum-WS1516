\section{Auswertung}
\label{sec:Auswertung}
\subsection{Bestimmung der Dichte}
Um die nachfolgend berechneten Elastizitätsmoduln mit Literaturwerten vergleichen
zu können, wird zunächst die Dichte der Stäbe bestimmt. Mit Hilfe einer Tabelle
wird dann das Metall oder die Legierung bestimmt.
Für Stab 1, der eine quadratische Querschnittsfläche hat, wurden folgende
geometrische Abmessungen bestimmt:
\begin{align*}
  \text{Länge } l_1 &= \SI{0.60}{\meter}\\
  \text{Breite } a &= \SI{0.01}{\meter}\\
  \text{Masse } m_1 &= \SI{0.5025}{\kilo\gram}.\\
  \intertext{Für Stab 2 mit einer runden Querschnittsfläche wurde}
  \text{Länge } l_2 &= \SI{0.60}{\meter}\\
  \text{Durchmesser } d &= \SI{0.01}{\meter}\\
  \text{Masse } m_2 &= \SI{0.5025}{\kilo\gram}
\end{align*}
gemessen. Alle Messungen wurden dabei zehn Mal durchgeführt, um den Fehler des
Mittelwerts bestimmen zu können, da jedoch immer der gleiche Wert gemessen wurde,
ist der Fehler 0.
Die Dichte der Probestäbe wird mit
\begin{equation}
  \rho = \frac{m}{V}
  \label{eqn:dichte}
\end{equation}
bestimmt.
Für die gemessene Dichte des quadratischen Stabes folgt
\begin{equation*}
  \rho_1 = \frac{m_1}{l_1 a^2} = \frac{\SI{0.5025}{\kilo\gram}}{\SI{0.60}{\meter}
  (\SI{0.01}{\meter})^2} = \SI{8375}{\kilo\gram \per \cubic\meter}
  \label{eqn:dichte_stab1}
\end{equation*}
und für die des runden Stabes
\begin{equation*}
  \rho_2 = \frac{m_2}{\pi \left(\frac{d}{2}\right)^2 l_2}
  = \frac{\SI{0.3605}{\kilo\gram}}{\pi \left(\SI{0.005}{\meter}\right)^2 \SI{0.55}{\meter}}
  = \SI{8346}{\kilo\gram \per \cubic\meter}.
  \label{eqn:dichte_stab2}
\end{equation*}

Dies entspricht im Rahmen der relativen Fehler von $\Delta \rho_1 = \SI{0.3}{\percent}$
und $\Delta \rho_2 = \SI{0.6}{\percent}$ der literaturbekannten Dichte von Messing
(\cite[274]{geschke}, Tabelle 1: Einige Eigenschaften fester Stoffe) und legt nahe,
dass die Stäbe aus Messing sind. Für eine Messing-Legierung mit einer Zusammensetzung
von $\SI{60}{\percent}$ Kupfer und $\SI{38}{\percent}$ Zink ist dort eine Dichte
von $\rho_{Messing}=\SI{8400}{\kilo\gram \per \cubic\meter}$ angegeben.

\subsection{Flächenträgheitsmoment}
Für die Berechnung des Elastizitätsmoduls werden die Flächenträgheitsmomente der
verschiedenen Stäbe benötigt. Das Flächenträgheitsmoment $I$ wird mit der Gleichung
(xx) berechnet.
\subsubsection{quadratischer Stab}
Für den Stab mit quadratischer Querschnittsfläche und einer Kantenlänge von
$h = \SI{0.01}{\meter}$ ergibt sich
\begin{equation}
  I_1 = \int_{-\frac{h}{2}}^{\frac{h}{2}} \int_{-\frac{h}{2}}^{\frac{h}{2}}
  y^2 \symup{d}x \symup{d}y
  = h \cdot \left[\frac{1}{3} y^3\right]_{-\frac{h}{2}}^{\frac{h}{2}}
  = \frac{h^4}{12} = \SI{8.33e-10}{\meter\tothe{4}}.
  \label{eqn:I_quadratisch}
\end{equation}
\subsubsection{runder Stab}
Weiter ergibt sich für den Stab mit runder Querschnittsfläche und dem Radius
$R=\frac{d}{2}=\SI{0.005}{\meter}$ unter Verwendung der Polarkoordinaten
$
\begin{pmatrix}
  x \\
  y
\end{pmatrix}
=
\begin{pmatrix}
  r \sin\varphi \\
  r \cos\varphi
\end{pmatrix}
$
\begin{equation}
  \begin{split}
  I_2 &= \int_0^{2\pi} \int_0^R r \cdot r^2 \cos^2\!\varphi\, \symup{d}r \symup{d}\varphi
  =\frac{R^4}{4}\left[\frac{1}{2}(\varphi + \underbrace{\sin\varphi \cos\varphi}_{=0})\right]_0^{2\pi} \\
  &=\frac{\pi R^4}{4} = \SI{4.91e-10}{\meter\tothe{4}}
  \end{split}
  \label{eqn:I_rund}
\end{equation}


\subsection{Einseitige Einspannung}
\subsubsection{eckiger Stab}
In Tabelle \ref{tab:messung1} sind die Messwerte dargestellt, die zur Berechnung
der Durchbiegung $D(x)$ nach Gleichung (xx) benötigt werden. Der Abstand des
Messpunktes vom Einspannungspunkt ist $x$, $D_0(x)$ ist die Durchbiegung ohne
Belastung des Stabes und $D_M(x)$ die Durchbiegung des Stabes nach Anhängen
eines Gewichts der Masse $M_{G1} = \SI{1.0205}{\kilo\gram}$. Die effektive Länge
des Stabes vom Einspannungspunkt bis zum freien Ende des Stabes beträgt
$l_\text{eff} = \SI{0.49}{\meter}$.
\begin{table}
  \centering
  \begin{tabular}{c c c}
    \toprule
    x in \si{\centi\meter} & $D_0(x)$ in \si{\milli\meter} & $D_M(x)$ in \si{\milli\meter} \\
    \midrule
    48.0	&	-0.28	&	4.74 \\
    47.5	&	-0.26	&	4.69 \\
    47.0	&	-0.24	&	4.64 \\
    46.5	&	-0.13	&	4.58 \\
    46.0	&	-0.22	&	4.51 \\
    45.5	&	-0.20	&	4.45 \\
    45.0	&	-0.19	&	4.39 \\
    44.5	&	-0.19	&	4.32 \\
    44.0	&	-0.17	&	4.26 \\
    43.5	&	-0.14	&	4.20 \\
    43.0	&	-0.14	&	4.14 \\
    42.5	&	-0.11	&	4.08 \\
    42.0	&	-0.11	&	4.01 \\
    41.5	&	-0.09	&	3.94 \\
    41.0	&	-0.08	&	3.87 \\
    40.5	&	-0.07	&	3.81 \\
    40.0	&	-0.05	&	3.76 \\
    39.0	&	-0.01	&	3.65 \\
    38.0	&	0.04	&	3.55 \\
    37.0	&	0.08	&	3.44 \\
    36.0	&	0.14	&	3.35 \\
    35.0	&	0.18	&	3.28 \\
    34.0	&	0.23	&	3.16 \\
    33.0	&	0.28	&	3.07 \\
    32.0	&	0.32	&	2.97 \\
    31.0	&	0.38	&	2.90 \\
    30.0	&	0.44	&	2.82 \\
    28.0	&	0.55	&	2.68 \\
    26.0	&	0.69	&	2.56 \\
    24.0	&	0.82	&	2.45 \\
    22.0	&	0.96	&	2.36 \\
    20.0	&	1.10	&	2.28 \\
    15.0	&	1.43	&	2.14 \\
    10.0	&	1.84	&	2.19 \\
     5.0	&	0.60	&	0.70 \\
     2.5	&	0.85	&	0.82 \\
    \bottomrule
  \end{tabular}
  \caption{Messwerte zur Berechnung von $D(x)$ für den eckigen Stab 1 bei einseitiger
  Einspannung.}
  \label{tab:messung1}
\end{table}

Der Elaztizitätsmodul wird durch eine lineare Ausgleichsrechnung bestimmt. Dazu
wird $D(x)$ gegen $L x^2 - \frac{x^3}{3}$ aufgetragen und mit Python eine lineare
Regression der Form $f(x) = a \cdot x + b$ durchgeführt, welche für die Parameter
$a$ und $b$ die folgenden Werte liefert:
\begin{align*}
  a &= \SI{0.06566(23)}{\per\squared\meter} \\
  b &= \SI{0.57(12)e-3}{\meter}.
\end{align*}
Nach Gleichung (xx) wird der Elastizitätsmodul dann mit
\begin{equation}
  a = \frac{F}{2 E I} \iff E = \frac{F}{2 I a}
  \label{eqn:Emodul1}
\end{equation}
bestimmt. Dabei ist $I$ das Flächenträgheitsmoment. Für den

\subsubsection{runder Stab}
\begin{table}
  \centering
  \begin{tabular}{c c c}
    \toprule
    x in \si{\centi\meter} & $D_0(x)$ in \si{\milli\meter} & $D_M(x)$ in \si{\milli\meter} \\
    \midrule
    48.0 & 1.70 & 6.25 \\
    47.5 & 1.60 & 6.17 \\
    47.0 & 1.60 & 6.09 \\
    46.5 & 1.58 & 6.01 \\
    46.0 & 1.58 & 5.92 \\
    45.5 & 1.56 & 5.82 \\
    45.0 & 1.55 & 5.73 \\
    44.5 & 1.53 & 5.65 \\
    44.0 & 1.52 & 5.57 \\
    43.5 & 1.51 & 5.48 \\
    43.0 & 1.49 & 5.40 \\
    42.5 & 1.49 & 5.33 \\
    42.0 & 1.48 & 5.26 \\
    41.5 & 1.47 & 5.17 \\
    41.0 & 1.45 & 5.09 \\
    40.5 & 1.44 & 5.01 \\
    40.0 & 1.44 & 4.94 \\
    39.0 & 1.43 & 4.77 \\
    38.0 & 1.42 & 4.64 \\
    37.0 & 1.41 & 4.50 \\
    36.0 & 1.41 & 4.38 \\
    35.0 & 1.40 & 4.24 \\
    34.0 & 1.40 & 4.11 \\
    33.0 & 1.40 & 3.98 \\
    32.0 & 1.40 & 3.85 \\
    31.0 & 1.42 & 3.75 \\
    30.0 & 1.43 & 3.63 \\
    28.0 & 1.48 & 3.42 \\
    26.0 & 1.51 & 3.25 \\
    24.0 & 1.55 & 3.06 \\
    22.0 & 1.61 & 2.91 \\
    20.0 & 1.67 & 2.75 \\
    15.0 & 1.84 & 2.50 \\
    10.0 & 2.08 & 2.39 \\
    5.0  & 0.61 & 0.69 \\
    2.5  & 0.75 & 0.78 \\
    \bottomrule
  \end{tabular}
  \caption{Messwerte zur Berechnung von $D(x)$ für den runden Stab 2 bei einseitiger
  Einspannung.}
  \label{tab:messung2}
\end{table}

\subsection{Beidseitige Einspannung}

\begin{table}
  \centering
  \begin{tabular}{c c c}
    \toprule
    x in \si{\centi\meter} & $D_0(x)$ in \si{\milli\meter} & $D_M(x)$ in \si{\milli\meter} \\
    \midrule
    26.6 & 1.44 & 3.52 \\
    26.1 & 1.45 & 3.52 \\
    25.6 & 1.44 & 3.50 \\
    25.1 & 1.43 & 3.49 \\
    24.6 & 1.42 & 3.47 \\
    24.1 & 1.42 & 3.46 \\
    23.6 & 1.42 & 3.44 \\
    23.1 & 1.41 & 3.41 \\
    22.6 & 1.41 & 3.39 \\
    22.1 & 1.40 & 3.37 \\
    21.1 & 1.39 & 3.31 \\
    20.1 & 1.37 & 3.24 \\
    19.1 & 1.36 & 3.27 \\
    18.1 & 1.34 & 3.09 \\
    17.1 & 1.32 & 3.00 \\
    16.1 & 1.30 & 3.10 \\
    15.1 & 1.28 & 2.80 \\
    10.1 & 1.17 & 2.27 \\
    5.1  & 1.08 & 1.66 \\
    2.5  & 1.11 & 1.42 \\
    \bottomrule
  \end{tabular}
  \caption{Messwerte zur Berechnung von $D(x)$ für den rechteckigen Stab 1 bei
  beidseitiger Einspannung.}
  \label{tab:messung3a}
\end{table}

\begin{table}
  \centering
  \begin{tabular}{c c c}
    \toprule
    x in \si{\centi\meter} & $D_0(x)$ in \si{\milli\meter} & $D_M(x)$ in \si{\milli\meter} \\
    \midrule
    28.6 & 3.15 & 5.23 \\
    29.1 & 3.15 & 5.22 \\
    29.6 & 3.15 & 5.22 \\
    30.1 & 3.17 & 5.23 \\
    30.6 & 3.18 & 5.22 \\
    31.1 & 3.18 & 5.22 \\
    31.6 & 3.18 & 5.22 \\
    32.1 & 3.19 & 5.21 \\
    32.6 & 3.20 & 5.21 \\
    33.1 & 3.21 & 5.19 \\
    34.1 & 3.22 & 5.17 \\
    35.1 & 3.25 & 5.14 \\
    36.1 & 3.26 & 5.10 \\
    37.1 & 3.29 & 5.05 \\
    38.1 & 3.29 & 5.00 \\
    39.1 & 3.30 & 4.94 \\
    40.1 & 3.31 & 4.88 \\
    45.1 & 3.37 & 4.50 \\
    50.1 & 3.36 & 3.99 \\
    52.5 & 3.36 & 3.70 \\
    \bottomrule
  \end{tabular}
  \caption{Messwerte zur Berechnung von $D(x)$ für den rechteckigen Stab 1 bei
  beidseitiger Einspannung.}
  \label{tab:messung3b}
\end{table}
