\section{Diskussion}
\label{sec:Diskussion}
Zunächst sei angemerkt, dass die Messuhren sehr anfällig auf Stöße auf den Tisch
reagiert haben und häufig geschwankt haben.
Bei der linearen Ausgleichsrechnung für den Bereich $0 \leq x \leq \frac{L}{2}$
der beidseitigen Einspannung fallen zwei Messwerte auf, die deutlich neben der
Ausgleichsgeraden liegen. Da alle anderen Messwerte sehr gut mit der Ausgleichsgeraden
übereinstimmen, könnte es sein, dass dort nicht darauf geachtet wurde, dass vor
dem Ablesen der Messuhr auf den Tisch geklopft werden musste, damit sich der
Zeiger richtig einstellen kann.
Werden die Werte für $x=\SI{16.1}{\centi\meter}$ und $x=\SI{19.1}{\centi\meter}$
nicht in die Regression mit einbezogen, so ergibt sich für den Elastitzitätsmodul
statt $E = \SI{95(2)e-9}{\newton\per\squared\meter}$ $E = \SI{94.4(3)e-9}
{\newton\per\squared\meter}$. Dies entspricht einer Abweichung um etwa $\SI{-0.2}
{\percent}$, vor allem aber wird der Fehler kleiner.
