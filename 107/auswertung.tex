\section{auswertung}
\label{auswertung}
Die Aus der Vesuchsanleitung entnommenden Daten sind die Falllänge $\Delta x$
die Apparaturkonstante der Kleinen Kugel $K_{kl}$ und die
aus \ref{} entnommende Dichte von Wasser \roh_f sind
\begin{align}
L=(0.1,0.0001)\si{\meter}\\
\roh_f=(0.998,0.001)\si{\kilo\grammg/\meter^(-3)}\\
K_kl=(0.07640\pm0.00001)\si{\frac{\meter\pascal\centi\meter^3}{\gramm}}
\end{align}
In der Tabelle \ref{}, sind die gemessenden Werte für Fallzeit, Durchmesser der Kugeln angegeben
\begin{table}
  \centering
  \begin{tabular}{c c c c}
    \toprule
    Durchmesser & Fallzeit & Durchmesser & Fallzeit\\
    $d_{kl}$ in $\si{\meter}$&$t_{kl}$in $\si{\second}$&$d_{gr}$ in $\si{\meter}$& $t_{gr}$ in $\si{\second}$\\
    \midrule
    15.627\pm0.001  &  12.76\pm0.01  &  15.795\pm0.001  &  87.33\pm0.01
    15.624\pm0.001  &  13.05\pm0.01  &  15.794\pm0.001  &  87.03\pm0.01
    15.631\pm0.001  &  13.04\pm0.01  &  15.792\pm0.001  &  88.39\pm0.01
    15.630\pm0.001  &  12.91\pm0.01  &  15.799\pm0.001  &  88.36\pm0.01
    15.626\pm0.001  &  12.50\pm0.01  &  15.793\pm0.001  &  88.01\pm0.01
    15.626\pm0.001  &  12.53\pm0.01  &  15.798\pm0.001  &  88.32\pm0.01
    15.633\pm0.001  &  12.69\pm0.01  &  15.799\pm0.001  &  88.07\pm0.01
    15.630\pm0.001  &  12.64\pm0.01  &  15.798\pm0.001  &  88.43\pm0.01
    15.631\pm0.001  &  13.04\pm0.01  &  15.797\pm0.001  &  88.03\pm0.01
    15.630\pm0.001  &  12.95\pm0.01  &  15.802\pm0.001  &  88.23\pm0.01
    \bottomrule
  \end{tabular}
  \caption{}
  \label{fig:Messwertegedaempfteschwingung}
\end{table}
Die Dichte der Kugeln, die Mittelwerte der Durchmesser und Fallzeiten sind
\begin{align*}
  \roh_{kl}=(2226\pm5)\si{\frac{\kilo\gramm}{\meter^{-3}}}\\
  \roh_{gr}=(2398\pm5)\si{\frac{\kilo\gramm}{\meter^{-3}}}\\
  d_{kl}=&(0.0156288\pm0.0000003)\si{\centi\meter}\\
  d_{gr}=&(0.0157967\pm0.0000003)\si{\centi\meter}\\
  t_{kl}=&(12.8110\pm0.0003)\si{\second}\\
  t_{gr}=&(88.0200\pm0.0003)\si{\second}\;\;.
\end{align*}
Der nach Gleichung \eqref{} berechnete Wert der Viskosität $\eta$, die Apparaturkonstante der
großen Kugel $K_{gr}$ nach Gleichung \eqref{} und die Reynoldszahlen $Re$ nach Gleichung
\eqref{} sind
\begin{equation*}
  \eta=1202\pm5\si{\meter\pascal\second}
  K_{gr}=0.00975\pm0.00005
  Re_{gr}=0.01490\pm0.00007
  Re_{kl}=0.1013\pm0.0005
\end{equation*}
In Abbildung \ref{fig:Viskositaet} ist dieVikosität $\log(\eta)$
in abhängigkeit von der Temperatur dargestellt.
\begin{figure}
  \centering
  \includegraphics[width=0.78\textwidth]{Viskosität_temp.pdf}
  \caption{}
  \label{fig:Viskositaet}
\end{figure}
Die gemessenden Werte sind in Tabelle \ref{fig:viskositaet_tab} aufgelistet. Die Fit hat die Form
nach Gleichung \eqref{} die Form
\begin{equation}
  \log(\eta)=log(A)+\frac{B}{T}\;\;,
\end{equation}
mit den Fitparametern
\begin{align*}
  A=20.3428628\pm4.30050338)10^2\si{\meter\pascal\second}
  B=-177.147899\pm6.59368587)10^(-4)\si{\kelvin}
\end{align*}
\begin{table}
  \centering
  \begin{tabular}{c c c c}
    \toprule
    Temperatur & Fallzeit & Fallzeit & Viskosität\\
    $T$ in $\si{\kelvin}$&$t_{kl}$ in $\si{\second}$ & $t_{gr}$ in $\si{\second}$& $\eta$ in $\si{\meter\pascal\second}$ \\
    \midrule
   293.15\pm0.01  &  88.03\pm0.01  &  88.23\pm0.01  &  1.204\pm0.005
   302.65\pm0.01  &  68.90\pm0.01  &  68.23\pm0.01  &  0.936\pm0.004
   308.65\pm0.01  &  61.18\pm0.01  &  61.95\pm0.01  &  0.841\pm0.003
   313.15\pm0.01  &  56.80\pm0.01  &  56.66\pm0.01  &  0.775\pm0.003
   318.15\pm0.01  &  51.56\pm0.01  &  51.69\pm0.01  &  0.705\pm0.003
   323.65\pm0.01  &  47.29\pm0.01  &  47.69\pm0.01  &  0.649\pm0.003
   328.15\pm0.01  &  43.90\pm0.01  &  43.83\pm0.01  &  0.599\pm0.002
   333.15\pm0.01  &  40.61\pm0.01  &  40.80\pm0.01  &  0.556\pm0.002
   338.15\pm0.01  &  37.66\pm0.01  &  38.01\pm0.01  &  0.517\pm0.002
   343.15\pm0.01  &  35.49\pm0.01  &  35.21\pm0.01  &  0.483\pm0.002
    \bottomrule
  \end{tabular}
  \caption{}
  \label{fig:viskositaet_tab}
\end{table}
