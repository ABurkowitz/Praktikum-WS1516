\section{Auswertung}
\label{sec:Auswertung}
Die aus der Vesuchsanleitung entnommenden Daten sind die Falllänge $\Delta x$
die Apparaturkonstante der kleinen Kugel $K_{\text{kl}}$ und die
aus der Literatur \cite{geschke} entnommende Dichte von Wasser $\rho_{\text{f}}$ sind

\begin{align*}
  L=&\SI{0.1000(1)}{\meter}\\
  \rho_{\text{f}}=&\SI{0.998(1)}{\kilo\gram\per\cubic\meter}\\
  K_{\text{kl}}=&\SI{0.07640(1)}{\meter\pascal\centi\cubic\meter\per\gram}\;\;.
\end{align*}
In der Tabelle \ref{tab:Messung1}, sind die gemessenden Werte für Fallzeit $t$,
 Durchmesser $d$ der Kugeln angegeben.
\begin{table}
  \centering
  \begin{tabular}{c c c c}
    \toprule
    Durchmesser & Fallzeit & Durchmesser & Fallzeit\\
    $d_{\text{kl}}$ in $\si{\meter}$&$t_{\text{kl}}$in $\si{\second}$&$d_{\text{gr}}$ in $\si{\meter}$& $t_{\text{gr}}$ in $\si{\second}$\\
    \midrule
    15.627\pm0.001  &  12.76\pm0.01  &  15.795\pm0.001  &  87.33\pm0.01\\
    15.624\pm0.001  &  13.05\pm0.01  &  15.794\pm0.001  &  87.03\pm0.01\\
    15.631\pm0.001  &  13.04\pm0.01  &  15.792\pm0.001  &  88.39\pm0.01\\
    15.630\pm0.001  &  12.91\pm0.01  &  15.799\pm0.001  &  88.36\pm0.01\\
    15.626\pm0.001  &  12.50\pm0.01  &  15.793\pm0.001  &  88.01\pm0.01\\
    15.626\pm0.001  &  12.53\pm0.01  &  15.798\pm0.001  &  88.32\pm0.01\\
    15.633\pm0.001  &  12.69\pm0.01  &  15.799\pm0.001  &  88.07\pm0.01\\
    15.630\pm0.001  &  12.64\pm0.01  &  15.798\pm0.001  &  88.43\pm0.01\\
    15.631\pm0.001  &  13.04\pm0.01  &  15.797\pm0.001  &  88.03\pm0.01\\
    15.630\pm0.001  &  12.95\pm0.01  &  15.802\pm0.001  &  88.23\pm0.01\\
    \bottomrule
  \end{tabular}
  \caption{Gemessende Zeiten und Druchmesser der Kugeln.}
  \label{tab:Messung1}
\end{table}
Die Dichte der Kugeln, die Mittelwerte der Durchmesser und Fallzeiten sind
\begin{align*}
  \rho_{\text{kl}}=&\SI[per-mode=fraction]{2226(5)}{\kilo\gram\per\cubic\meter}\\
  \rho_{\text{gr}}=&\SI[per-mode=fraction]{2398(5)}{\kilo\gram\per\cubic\meter}\\
  d_{\text{kl}}=&\SI[per-mode=fraction]{0.0156288(3)}{\centi\meter}\\
  d_{\text{gr}}=&\SI[per-mode=fraction]{0.0157967(3)}{\centi\meter}\\
  t_{\text{kl}}=&\SI[per-mode=fraction]{12.8110(3)}{\second}\\
  t_{\text{gr}}=&\SI[per-mode=fraction]{88.0200(3)}{\second}\;\;.
\end{align*}
Der nach Gleichung \eqref{eqn:bestimmung_eta} berechnete Wert der Viskosität $\eta$, die Apparaturkonstante der
großen Kugel $K_{\text{gr}}$ nach umgestellter Gleichung \eqref{eqn:bestimmung_eta} und die Reynoldszahlen $Re$ nach Gleichung
\eqref{eqn:reynoldszahl} sind
\begin{align*}
  \eta=&\SI{1.202(5)}{\meter\pascal\second}\\
  K_{\text{gr}}=&\SI[per-mode=fraction]{0.00975(5)}{\meter\pascal\centi\cubic\meter\per\gram}\\
  Re_{\text{gr}}=&(14.90\pm0.07)\\
  Re_{\text{kl}}=&(101.3\pm0.5)\;\;.
  \label{eqn:etaKRE}
\end{align*}
In Abbildung \ref{fig:Viskositaet} ist die Viskosität $\log(\eta)$
in Abhängigkeit von der Temperatur dargestellt.
\begin{figure}
  \centering
  \includegraphics[width=0.78\textwidth]{Viskosität_temp.pdf}
  \caption{Graphische Darstellung der Messwerte zur Temperaturabhängigkeit der Viskosität.}
  \label{fig:Viskositaet}
\end{figure}
Die gemessenden Werte sind in Tabelle \ref{fig:viskositaet_tab} aufgelistet. Die Fit hat
nach der umgestellten Gleichung \eqref{eqn:andradesche_gleichung} die Form
\begin{equation}
  \ln(\eta)=\ln(A)+\frac{B}{T}\;\;,
\end{equation}
mit den Fitparametern
\begin{align*}
  A=&\SI{2.5(3)e-3}{\meter\pascal\second}\\
  B=&\SI[per-mode=fraction]{ 1.80(4)e3}{\kelvin}\;\;.
\end{align*}
\begin{table}
  \centering
  \begin{tabular}{c c c c c}
    \toprule
    Temperatur & Fallzeit & Viskosität & Werte der X-Achse & Werte der Y-Achse\\
    $T$ in $\si{\kelvin}$& $t_{\text{gr}}$ in $\si{\second}$& $\frac{\eta}{10^3}$ in $\si{\pascal\second}$ &
    $\frac{1}{T}$ in $\frac{1}{10^3\si{\kelvin}}$ & $\ln(\frac{\eta}{\si{\pascal\second}})$\\
    \midrule
   293.15\pm0.01  &  88.23\pm0.01  &  1.204\pm0.005  &  3.411 \pm0.001  &   6.722\pm0.004\\
   302.65\pm0.01  &  68.23\pm0.01  &  0.936\pm0.004  &  3.304 \pm0.001  &  -6.973\pm0.004\\
   308.65\pm0.01  &  61.95\pm0.01  &  0.841\pm0.003  &  3.239 \pm0.001  &  -7.081\pm0.004\\
   313.15\pm0.01  &  56.66\pm0.01  &  0.775\pm0.003  &  3.194 \pm0.001  &  -7.163\pm0.004\\
   318.15\pm0.01  &  51.69\pm0.01  &  0.705\pm0.003  &  3.1427\pm0.0010  &  -7.257\pm0.004\\
   323.65\pm0.01  &  47.69\pm0.01  &  0.649\pm0.003  &  3.0893\pm0.0010  &  -7.341\pm0.004\\
   328.15\pm0.01  &  43.83\pm0.01  &  0.599\pm0.002  &  3.0469\pm0.0009  &  -7.420\pm0.004\\
   333.15\pm0.01  &  40.80\pm0.01  &  0.556\pm0.002  &  3.0012\pm0.0009  &  -7.495\pm0.004\\
   338.15\pm0.01  &  38.01\pm0.01  &  0.517\pm0.002  &  2.9568\pm0.0009  &  -7.568\pm0.004\\
   343.15\pm0.01  &  35.21\pm0.01  &  0.483\pm0.002  &  2.9138\pm0.0008  &  -7.636\pm0.004\\
    \bottomrule
  \end{tabular}
  \caption{Fallzeiten und Viskositäten bei verschiedenen Temperaturen, sowie die
  die Wertepaare für Abbildung \ref{fig:Viskositaet}.}
  \label{fig:viskositaet_tab}
\end{table}
