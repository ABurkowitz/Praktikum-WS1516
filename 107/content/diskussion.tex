\section{Diskussion}
\label{sec:Diskussion}
Da die Reynoldzahlen bei beiden Kugeln deutlich kleiner sind als der
Wert $Re=1160$ aus der Literatur \cite{gertzen}, sind sie mit Sicherheit laminar.
Der berechnete Wert der Viskosität bei Raumtemperatur
weicht aus unbekannten Gründen vom Literaturwert $\eta=\SI{1.005}{\meter\pascal\second}$\cite{gertzen}
 um $20\%$ ab.
Die Raumtemperatur wurde aber nicht genau bestimmt, was jedoch nur einen Teil der
Abweichung erklärt, da ein vergleichbarer Wert erst bei $13\si{\celsius}$ erreicht wird.
Eine weitere Erklärung ist, das die Viskosität logarithmisch zur Temperatur abfällt
und somit der Fehler sich auch logarithmisch ändert.
Die kleinen Fehler bei der Viskosität und der Apparaturkonstante zeigen, das
es sich um eine genaue Messmethode handelt. Die geringe Abweichung des Fits von
den Messwerten bestätigt dies. Die Viskosität sinkt wie erwartet mit steigender
Temperatur.
