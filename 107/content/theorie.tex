\section{Theorie}
\label{sec:Theorie}
Auf jeden Körper, der sich durch eine Flüssigkeit bewegt, wirkt eine
geschwindigkeitsabhängige Reibungskraft $\vec{F_\text{R}}$, welche zusätzlich
von der Berührungsfläche $A$ abhängt. Fällt eine Kugel in einer zähen Flüssigkeit,
so wirken neben der Reibungskraft noch die Schwerkraft $\vec{F_\text{G}}$ und
der Auftrieb $\vec{F_\text{A}}$. Dabei sind die Reibungskraft und der Auftrieb
der Schwerkraft entegegengerichtet. Mit zunehmender Zeit nimmt die Fallgeschwindigkeit
und damit die Reibungskraft so lange zu, bis sich ein Kräftegleichgewicht einstellt.
Der Körper fällt dann mit konstanter Geschwindigkeit weiter.
\cite{sample}
