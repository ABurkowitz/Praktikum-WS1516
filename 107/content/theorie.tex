\section{Theorie}
\label{sec:Theorie}
Auf jeden Körper, der sich durch eine Flüssigkeit bewegt, wirkt eine
geschwindigkeitsabhängige Reibungskraft $\vec{F_\text{R}}$, welche zusätzlich
von der Berührungsfläche $A$ abhängt. Fällt eine Kugel in einer zähen Flüssigkeit,
so wirken neben der Reibungskraft noch die Schwerkraft $\vec{F_\text{G}}$ und
der Auftrieb $\vec{F_\text{A}}$. Dabei sind die Reibungskraft und der Auftrieb
der Schwerkraft entegegengerichtet. Mit zunehmender Zeit nimmt die Fallgeschwindigkeit
und damit die Reibungskraft so lange zu, bis sich ein Kräftegleichgewicht einstellt.
Der Körper fällt dann mit konstanter Geschwindigkeit weiter.
Massgeblich für das Bewegungsverhalten hierbei ist die dynamische Viskosität
$\eta$, welche eine Materialkonstante der Flüssigkeit ist und stark von der
Temperatur abhängt. Sie wird durch die Andradesche Gleichung
\begin{equation}
  \eta(T) = A \exp{\frac{B}{T}}
  \label{eqn:andradesche_gleichung}
\end{equation}
mit den Konstanten $A$ und $B$ und der Temperatur $T$ beschrieben und wird als
dynamisch bezeichnet, da sie bei Temperaturänderung nicht konstant bleibt.
Die Viskosität eines Fluids kann mit einem Kugelfall-Viskosimeter bestimmt werden.
Dabei fällt eine Kugel mir dem Radius $r$ in einem mit einer Flüssigkeit
gefüllten Rohr, wobei darauf geachtet werden muss, dass die Strömung laminar ist,
sich also keine Wirbel bilden. Um genau feststellen zu können, ob eine Strömung
laminar oder turbulent ist, muss die Reynoldszahl
\begin{equation}
  Re = \frac{\rho v d}{\eta}
  \label{eqn:reynoldszahl}
\end{equation}
betrachtet werden. Dabei ist $\rho$ die Dichte der Flüssigeit, $v$ die
Strömgeschwindigkeit, also die Geschwindigkeit der Kugel und $d$ eine charakteristische
Länge, hier beispielsweise der Durchmesser des Rohres, in dem die Kugel fällt.
Ist die Reynoldszahl kleiner als die kritische Reynoldszahl
\begin{equation*}
  Re_{krit} \approx 2300,
\end{equation*}
so ist die Strömung laminar. Bei einer laminaren Strömung lässt sich die innere
Reibung durch die Stokessche Reibung
\begin{equation}
  F_\text{R} = 6 \pi \eta v r
  \label{eqn:stokes}
\end{equation}
beschreiben.

Beim Kugelfallviskosimeter nach Höppler fällt eine Kugel durch ein Rohr mit einem
nur wenig größerem Radius als der Radius der Kugel. Damit die Kugel nicht
schlingert und sich Turbulenzen bilden, wird das Rohr leicht geneigt, sodass die
Kugel an der Rohrwand entlanggleiten kann. Die Viskosität lässt sich dann mit
\begin{equation}
  \eta = K (\rho_\text{K} - \rho_\text{Fl}) \cdot t
  \label{eqn:bestimmung_eta}
\end{equation}
bestimmen, wenn die Fallzeit $t$, die Apparaturkonstante $K$, welche sowohl die
Fallhöhe als auch die Kugelgeometrie berücksichtig, und die Dichte $\rho_\text{K}$
der Kugel und die Dichte $\rho_\text{Fl}$ der Flüssigkeit bekannt sind.
\cite{sample}
