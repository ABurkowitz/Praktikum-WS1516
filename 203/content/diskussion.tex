\section{Diskussion}
\label{sec:Diskussion}
Die Messwerte im Niedrigdruckbereich und der Fit in Abbildung \ref{fig:fit1} stimmen bis auf kleine
Abweichungen überein. Auch im Hochdruckberich stimmen Messwerte und Fit in
Abbildung \ref{fig:fit2} überein. Wie erwartet steigt der Dampfdruck mit
höheren Temperaturen, da die einzelnen Moleküle im durchschnittlich mehr
Energie haben. Abweichungen in den beiden Messreihen lassen sich dadurch erklären,
dass sich nicht immer ein Gleichgewicht eingestellt hat, wenn die Messwerte 
abgelesen werden. Ein Anstieg des
Innendrucks der Apparatur nach dem Schließen des Absperrhahns zeigt, dass diese
nicht ganz dicht war, was auch zu Messungenauigkeiten geführt haben kann.  Wenn man die
Temperaturabhägigkeit der Verdampfungswärme betrachte wird klar, dass nur die
Variante mit der positiven Wurzel aus Abbildung \ref{fig:fit3} sinn macht, da
die Verdampfungswärme beim Kritischenpunkt null sein muss.  Nur so ist es möglich,
das Flüssigkeits- und Gasphase gleichzeit koexistieren. Auch ist die Verdampfungwärme
 in Abbildunge \ref{fig:fit4} für die Temperaturen zu klein. Auch ist ein konvexer
 nicht sinnvoll.
