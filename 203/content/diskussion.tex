\section{Diskussion}
\label{sec:Diskussion}
Die Fit Parameter der ersten Messung haben einen kleinen Fehler, der Weniger als
$5\%$ aus macht. Die Parameter zum Fit der zweiten Messung haben relativ
große Fehler von bis zu 50\%. Wie erwartet steigt der Dampfdruck mit
höheren Temperaturen, da die einzelnen Moleküle im Durchschnitt mehr
Energie haben. Abweichungen in den beiden Messreihen lassen sich dadurch erklären,
dass sich nicht immer ein Gleichgewicht eingestellt hat, wenn die Messwerte
abgelesen werden. Ein Anstieg desInnendrucks der Apparatur nach dem Schließen des
Absperrhahns zeigt, dass diese nicht ganz dicht war, was auch zu Messungenauigkeiten
geführt haben kann. Da die innere Verdampfungswärme viel größer als die äußere
Verdampfungswärme ist, lässt sich daraus schließen, dass die Überwindung der molekularen
Bindungen den Großteil der Verdampfungswärme aus macht. Wenn man die
Temperaturabhägigkeit der Verdampfungswärme betrachtet wird klar, dass nur die
Variante mit der positiven Wurzel aus
Abbildung \ref{fig:fit3} Sinn ergibt, da die Verdampfungswärme beim kritischen
Punkt null sein muss.  Nur so ist es möglich, das Flüssigkeits- und Gasphase
gleichzeit koexistieren. Auch weicht der Wert für die Verdampfungwärme in
Abbildung \ref{fig:fit4} zu stark ab.
