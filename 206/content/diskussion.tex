\newpage
\section{Diskussion}
\label{sec:Diskussion}
Bei der Messung gab es eine Abweichung der Messwerte für $T_1$. Die Werte für
Minute 8 und Minute 9 nach Beginn der Messung sind leicht erhöht, was daran
liegen könnte, dass der Rührer in Reservoir 1 defekt war und nur in unregelmäßigen
Abständen funktioniert hat, sodass dadurch keine gleichmäßige Verteilung möglich war.
Bei dem Berechnen der Güteziffern ist aufgefallen, dass die reale Güteziffer nur
ein Bruchteil der Idealen ist. Das liegt daran, das die Isolierung mangelhaft ist
und der Prozess nicht vollkommen reversibel ist wie für eine ideale Wärmepumpe
gefordert. Die mangelhafte Isolierung und die damit beeinflussten Messwerte,
beeinträchtigten die Genauigkeit der berechneten Werte.
