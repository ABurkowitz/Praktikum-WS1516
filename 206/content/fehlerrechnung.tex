\section{Fehlerrechnung}
\label{sec:Fehlerrechnung}
Der Mittelwert kann mit
\begin{equation}
  v_i= \frac{1}{n}\sum\nolimits_{i=1}^n x_i
  \label{eqn:gl11}
\end{equation}
berechnet werden. Die standardabweichung mit der Formel
\begin{equation}
  \label{eqn:gl12}
  s_1= \sqrt{\frac{1}{N-1}\sum\nolimits_{j=1}^N(v_j-v_i)^2} .
\end{equation}

\begin{equation}
  \label{eqn:gl13}
  \sigma_i=\frac{s_i}{\sqrt{N}}=
  \frac{\sqrt{\frac{1}{N-1}\sum\nolimits_{j=1}^N(v_j-\bar{v_i})^2}}{\sqrt{N}}
\end{equation}
\begin{equation}
  \Delta x_k = \frac{df}{dk} \cdot \sigma_k
\end{equation}
\begin{equation}
  \Delta x_{k,rel}=1 \pm \frac{\Delta x_k}{|x|}*100 \%
\end{equation}
\begin{equation}
  \Delta x_k = \frac{df}{dk} \cdot \sigma_k
\end{equation}
\begin{equation}
  \sigma^2=\sum\nolimits_{k=1}^n [y_k-(\bar{B}x_k+\bar{A})]^2
\end{equation}
