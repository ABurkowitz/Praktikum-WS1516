\section{Theorie \cite{sample}}
\label{sec:Theorie}
  Nach dem zweiten Hauptsatz der Themodynamik, der besagt, dass die Entropie in
  einem abgeschlossenen System niemals abnehmen kann, verläuft ein Wärmeaustausch
  zwischen zwei Reservoiren unterschiedlicher Temperatur immer vom wärmeren zum
  kälteren hin.

  Es ist jedoch möglich die Richtung des Wärmetransports umzukehren, wenn man
  dem System Energie in Form von mechanischer Arbeit zuführt. Ist dies der Fall,
  so spricht man von einer Wärmepumpe.

  Aus dem Verhältnis der aufzuwendenden Arbeit A und der an das wärmere Reservoir
  abgegebenen Wärmemenge $Q_1$ resultiert die Güteziffer $v$ einer Wärmepumpe.

  Da nach dem ersten Hauptsatz der Themodynamik die totale Energie ein einem
  abgeschlossenen System erhalten bleiben muss, muss die abgegebene Wärmemenge
  $Q_1$ gleich der Summe aus der aufgewandten Arbeit und der aus dem kälteren
  Reservoir entnommenen Wärmemenge $Q_2$ sein:
  \begin{equation}
    Q_1 = Q_2 + A
  \end{equation}
