\section{Auswertung}
\label{sec:Auswertung}
\subsection{Strahlungsleistung als Funktion der Temperatur und Bestimmung des Emissionsvermögens}
Die zu Beginn des Versuchs gemessene Offsetspannung beträg $U_0=0,0062 \,\si{\milli\volt}$.
Da die Ansprechzeit der verwendeten Geräte sehr gering ist, entfällt die Messung
der Anprechzeit der Themosäule.
Es werden die Thermospannungen für die vier Oberflächen des Leslie-Würfels als
Funktion der Temperatur gemessen, nachdem die Temperatur von der Anfangstemperatur
$T_\symup{max}=365,85 K$ um je weitere $5 K$ gefallen ist.
\begin{table}
  \centering
  \begin{tabular}{c c c c c}
    \toprule
    Temperatur T in K & \multicolumn {4}{c}{Thermospannung U in \si{\milli\volt}}\\
    & schwarz & glänzend & \;weiß & \quad matt \\
    \midrule
     365.85 & 1.031 & 0.072 & \; 0.992 & \quad 0.172 \\
     360.85 & 0.958 & 0.071 & \; 0.895 & \quad 0.156 \\
     355.85 & 0.856 & 0.065 & \; 0.816 & \quad 0.151 \\
     350.85 & 0.761 & 0.060 & \; 0.732 & \quad 0.138 \\
     345.85 & 0.666 & 0.051 & \; 0.639 & \quad 0.107 \\
     340.85 & 0.580 & 0.039 & \; 0.552 & \quad 0.101 \\
     335.85 & 0.503 & 0.031 & \; 0.476 & \quad 0.081 \\
     330.85 & 0.428 & 0.028 & \; 0.408 & \quad 0.069 \\
     325.85 & 0.353 & 0.030 & \; 0.333 & \quad 0.064 \\
     320.85 & 0.278 & 0.019 & \; 0.271 & \quad 0.058 \\
     315.85 & 0.210 & 0.018 & \; 0.199 & \quad 0.039 \\
     310.85 & 0.149 & 0.019 & \; 0.145 & \quad 0.036 \\
    \bottomrule
  \end{tabular}
  \caption{Thermospannungen in Abhänigkeit der Temperatur und Oberfläche.}
\end{table}

Die erneut gemessene Offestspannung ergibt nach Ende der Messreihe einen Wert von
$U_0=0.0081 \,\si{\milli\volt}$. Der Mittelwert der Offsetspannung, der nach
\begin{equation}
  \bar{x} = \frac{1}{N} \sum_{\symup{i}=1}^{N} x_\symup{i}
  \label{eqn:mittelwert}
\end{equation}
berechnet wird, ergibt sich zu $\bar{U_0} = 0.0072 \,\si{\milli\volt}$.

Die Thermospannung wird als Funktion von $T^4 - T_0^4$ aufgetragen, wobei $T_0 = 294,15 K$
die Raumtemperatur ist.
\begin{figure}
  \centering
  \includegraphics[width=0.82\textwidth]{thermospannung.pdf}
  \caption{Messwerte und Fits für die Thermospannung in Abhängigkeit von $T^4 - T_0^4$.}
  \label{fig:thermospannung}
\end{figure}
\newline
Die Fits werden mit linearer Regression der Form $f(x) = a \cdot x + b $ mit Python
berechnet.
Für die Parameter ergeben sich folgende Werte:
\begin{align*}
  a_\symup{schwarz}   &= \,(1.047\pm0.007)\cdot10^{-10} \si{\frac{\volt}{\kelvin^4}} \\
  b_\symup{schwarz}   &= (-0.836\pm0.010) \si{\volt} \\
  a_\symup{glaenzend} &= \,(7.3\pm0.5)\cdot10^{-12} \si{\frac{\volt}{\kelvin^4}} \\
  b_\symup{glaenzend} &= (-0.062\pm0.006) \si{\volt} \\
  a_\symup{weiss}     &= \,(9.95\pm0.06)\cdot10^{-11} \si{\frac{\volt}{\kelvin^4}} \\
  b_\symup{weiss}     &= (-0.794\pm0.008) \si{\volt} \\
  a_\symup{matt}      &= \,(1.67\pm0.07)\cdot10^{-11} \si{\frac{\volt}{\kelvin^4}} \\
  b_\symup{matt}      &= (-0.131\pm0.009) \si{\volt}
\end{align*}
Es wird nun angenommen, dass die schwarze Oberfläche ein schwarzer Strahler ist
und demnach $\epsilon_\symup{schwarz} = 1$ gilt.
Dann lassen sich die Emissionsvermögen der anderen Oberflächen wie folgt berechnen:
\begin{equation}
  \epsilon_\symup{i} = \frac{a_\symup{i}}{a_\symup{schwarz}},
  \label{eqn:epsilon}
\end{equation}
und man erhält als Ergebnis für die Emissionsvermögen $\epsilon_\symup{i}$:
\begin{align*}
  \epsilon_\symup{schwarz}   &= 1 \\
  \epsilon_\symup{glaenzend} &= (0.069\pm0.004) \\
  \epsilon_\symup{weiss}     &= (0.951\pm0.009) \\
  \epsilon_\symup{matt}      &= (0.159\pm0.006).
\end{align*}
Der Fehler berechnet sich dabei nach der Gaußschen Fehlerfortpflanzung
\begin{equation}
  \Delta f(x_\symup{i}) = \sqrt{\sum_{\symup{i}=1}^N
  \left(\frac{\partial f}{\partial x_\symup{i}}\right)^2 \cdot
  (\Delta x_\symup{i})^2}
\end{equation}
für Funktionen mit den fehlerbehafteten Größen $\Delta x_\symup{i}$.


\subsection{Strahlungsleistung als Funktion des Abstands}
In einer weiteren Messreihe werden die Thermospannungen bei konstanter Temperatur
$T=320,25\si{\kelvin}$ und veränderlichem Abstand für die schwarze Oberfläche
gemessen. Die Ergebnisse sind in nachfolgender Tabelle dargestellt.
\begin{table}
  \centering
  \begin{tabular}{c c}
    \toprule
    Abstand x in cm & Thermospannung U in \si{\milli\volt}\\
    \midrule
    10 & 0.269  \\
    11 & 0.264  \\
    12 & 0.260  \\
    13 & 0.2530 \\
    14 & 0.2474 \\
    15 & 0.2426 \\
    16 & 0.2373 \\
    17 & 0.2301 \\
    18 & 0.2236 \\
    19 & 0.2176 \\
    20 & 0.2108 \\
    \bottomrule
  \end{tabular}
 \caption{Thermospannungen der schwarzen Fläche\\in Abhänigkeit des Abstands.}
\end{table}
\newpage
Als grafische Darstellung erhält man
\begin{figure}

  \includegraphics[width=0.9\textwidth]{strahlintensitaet.pdf}
  \caption{Strahlenintensität in Abhängigkeit des Abstands.}
  \label{fig:strahlintensitaet}
\end{figure}
mit den Fitparametern c und d für die lineare Regression der Form $f(x) = c \cdot x + d $
\begin{align*}
  c &= (-0,00583\pm0.00011) \si{\frac{\milli\volt}{\centi\meter}} \\
  d &= (0,3289\pm0.0016) \si{\milli\volt}
\end{align*}
