\section{Diskussion}
\label{sec:Diskussion}
Zunächst sei angemerkt, dass alle Messungen der Thermospannung sehr anfällig auf
äußere Einflüsse, wie vorbeilaufende Personen oder die Körperwärme der Experimentatoren,
sind. Je größer der Abstand der Thermosäule zu dem Leslie-Würfel und je kleiner
die Strahlintensität und dementsprechend die gemessene Thermospannung, desto stärker
wirken sich die Einflüsse der Umgebung auf die Messwerte aus.
Zu Beginn der Messung der Thermospannung in Abhänigkeit von der Temperatur ist
die Temperatur während der Messungen um etwa $2 \si{\kelvin}$ gefallen. Da die Flächen
jedoch immer in der gleichen Reihenfolge gemessen wurden, hat dies keinen Einfluss
auf die einzelnen Temperaturverläufe, sondern nur auf die Vergleichbarkeit der
Emissionsvermögen.

Bei der Berechnung der Emissionsvermögen der einzelnen Flächen wurde die vereinfachende
Annahme getroffen, dass die schwarze Oberfläche einem schwarzen Strahler gleiche
und demnach einen Wert von $\epsilon_\symup{schwarz} = 1$ habe. Dieser Wert wird
in der Realität nicht erreicht und da die Berechnung der Emissionsvermögen der
anderen drei Oberflächen auf dieser Annahme basiert, werden auch diese Ergebnisse
verfälscht.

Dennoch kann man einige Erkenntnisse gewinnen:\newline
Die schwarze Fläche hat, wie zu erwarten war, das höchste Emissionsvermögen und
wegen \eqref{eqn:verhaeltnisse} auch das höchste Absorptionsvermögen, jedoch dafür das geringste
Reflektionsvermögen.
Mit $\epsilon_\symup{weiss} = (0.951\pm0.009)$ hat die weiße Oberfläche ebenfalls
ein sehr gutes Emissions- und Absorptionsvermögen und ein eher geringes Reflektionsvermögen.
Beide Metalloberflächen haben ein deutlich schlechteres Emissions- und Absorptionsvermögen,
was damit zusammenhängt, dass sie ein besseres Reflektionsvermögen haben.
Dabei hat die glänzende Oberfläche mit $\epsilon_\symup{glaenzend} = (0.069\pm0.004)$
ein noch geringeres Emissionsvermögen als die matte Oberfläche mit
$\epsilon_\symup{matt} = (0.159\pm0.006)$.

Das Ergebnis der Messung der Strahlintensität in Abhänigkeit des Abstandes ist
überraschend: Die Messwerte deuten auf einen linearen Abfall hin, obwohl eine
$\frac{1}{x^2}$-Abhängigkeit wegen der Abstrahlung in alle Raumrichtungen zu
erwarten gewesen wäre. Möglicherweise liegt dies daran, dass mit steigendem Abstand
die äußeren Einflüsse wie oben beschrieben zunehmen.
