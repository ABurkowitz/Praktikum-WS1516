\section{Durchführung}
\label{sec:Durchführung}
\subsection{Versuchsaufbau}
Es wird ein Strahlungswürfel nach Leslie auf einer Schiene montiert, sowie eine
Thermosäule nach Moll. Die Thermosäule misst die Wärmestrahlung relativ zur
Gehäusetemperatur. Der Würfel wird mit möglichst heißem Wasser befüllt. Die
Thermosäule wird so positioniert, dass sie $10\si{\centi\meter}$ von einer
Seitenfläche des Strahlungswürfels entfernt ist. Dabei wird das Experiment an
einem Ort durchgeführt, an dem die Wärmestrahlung durch Außeneinflüsse, wie
Sonnenstrahlung und Körperwärme des Experimentators, möglichst gering und konstant
gehalten wird. Mit einem Thermometer wird die Temperatur des Wassers im Inneren
des Strahlungswürfels gemessen. Mit einem Schutzfenster wir die spektrale
Empfindlichkeit von $\lambda=(0.2-50)\mu m$ auf $\lambda=(0.3-2.8)\mu m$ reduziert.

\subsection{Versuchsdurchführung}
Es wird eine Schwarze Pappe vor die Thermosäule gehalten und die Offsetspannung
$U_0$ gemessen.
Bei einer Temperatur nahe $373.15\si{\kelvin}$ wird die Thermospannung von den vier
Seiten des Strahlungswürfels gemessen und dann in einem Abstand von $5\si{\kelvin}$ die
Messung wiederholt. Am Ende der Messung wird erneut die Offsetspannung bestimmt,
um die Veränderung in die Ergebnisse einzurechnen. Die Ergebnisse werden als
Funktion von $T^4-T_0^4$ dargestellt und das Emissionsvermögen der Oberflächen
des Strahlungswürfels betimmt wobei die schwarz lakierte Seite als
Schwarzer Strahler betrachtet wird.
Die Strahlenintensität in Abhängigkeit von dem Abstand wird gemessen und
graphisch dargestellt.
