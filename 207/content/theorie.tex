\section{Theorie}
\label{sec:Theorie}
Alle Körper geben Wärmestrahlung ab. Ihr Reflexionsvermögen, Emissionsvermögen und
Absorbtionsvermögen stehen im folgenden Verhältnis
\begin{equation}
  \epsilon(\lambda,T)=A(\lambda,T)=1-R(\lambda,T).
  \label{eqn:verhaeltnisse}
\end{equation}
Schwarze Körper sind Objekte die sämtliche auf sie treffende Strahlung aufnehmen
\mbox{($\epsilon=1$)} und in Wärme umwandeln. Solche Objekte existieren in der Realität
nicht. Körper, die einem Schwarzen Körper am nächsten kommen, sind Hohlräume mit
einem kleinen Loch, da die auftreffende Strahlung im Innern solange reflektiert
wird, bis die gesamte Strahlung absorbiert ist und nur geringe Mengen durch das
Loch entweichen. Ist das Emissionsvermögen eines Körpers $\epsilon< 1$, wird
dieser als Grauer Körper bezeichnet. Das Planksche Strahlungsgesetz gibt die
abgestrahlte Leistung in Abhängigkeit von der Wellenlänge der Strahlung und der
Temperatur an.
\begin{equation}
  P(\lambda,T)=\frac{dP}{d\lambda}=\frac{2\pi c^2h}{\Omega_0\lambda^5}
  \left(\exp{ \left(\frac{c\,h}{k\,\lambda\,T}\right)} -1 \right)^{-1}
\end{equation}
Hierbei ist $h$ das Planksche Wirkungsquantum $k$ die Boltzmann-Konstante
und $\Omega_0$ beschreibt den Raumwinkel der abgestrahlten Wärme. Die wahrscheinlichste
Wellenlänge der Strahlung wird mit zunehmender Temperatur kleiner. Das
Stefan Boltzmann-Gesetz gibt die integrale Strahlungsdichte $P(T)$ eines schwarzen
Körpers an und ist gegeben durch
\begin{equation}
  P(T)=\epsilon \, \sigma \, T^4.
\end{equation}
\epsilon gibt dabei das Emissionsvermögen des Strahlers an während
$\sigma=5.67\cdot 10^{-8} \si{\frac{\watt}{\meter^2\kelvin^4}}$ die
Stefan-Boltzmann-Konstante ist.
\cite{sample}
