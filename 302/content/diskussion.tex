\section{Diskussion}
\label{sec:Diskussion}
Bei der Bestimmung der unbekannten Widerstände, Kapazitäten sowie Induktivitäten
ist zu beachten, dass nicht von allen Bauteilen der Fehler bekannt war. Zur Bestimmung
der unbekannten Fehler wurden die Bauteile, sofern möglich, dreimal variiert.
Um den Fehler genauer eingrenzen zu können, hätten die Bauteile jedoch viel öfter
variiert werden müssen.
Des Weiteren wurde bei den Messungen für $R_{11}, R_{14}, C_1 \text{ und } C_6$
vorausgesetzt, dass es sich um ideale Widerstände bzw. ideale Kondensatoren ohnde
Verluste handelt, was jedoch in der Realität nicht möglich ist.

Bei der Untersuchung der Wien-Robinson-Brückenschaltung wurde angenommen, dass $U_S$
im Frequenzverlauf nahezu konstant bleibt. Es wurde daher mit dem Mittelwert
$\Delta{U_S}=6.94 \si{\volt}$, der stichprobenartig bei verschieden großen Frequenzen
gemessenen Werte, gerechnet. Dadurch ergeben sich jedoch vor allem in den tiefen
und sehr hohen Frequenzen Abweichungen, denn dort beträgt die relative Abweichung
von $U_S$ zum Mittelwert in etwa 7\%, bei den mittleren Frequenzen nur etwa 2\%.

Die Klirrfaktormessung bestätigt das Ergebnis aus der Untersuchung der
Wien-Robinson-Brückenschaltung: In der Nähe der Sperrfrequenz $v_0$ stimmen die
Messwerte sehr gut mit der Theoriekurve überein, entfernt man sich weiter, wird
die Abweichung immer größer.
