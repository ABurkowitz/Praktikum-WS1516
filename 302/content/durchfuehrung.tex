\section{Durchführung}
\label{sec:Durchführung}
\subsection{Wheatstonesche Brücke}
Es wird eine Wheatonsche Brückenschaltung nach Abbildung\ref{fig:wheatonsche}
aufgebaut. Die Widerstände $R_3$ und $R_4$ werden so variiert, sodass das
Oszillioskop, das als Nullindikator dient, eine Minimale Spannung anzeigt.
Mit anderen Wiederständen $R_2$ wird die Messung zur Fehlerbestimmung wiederholt,
sowie mit einem anderen Widerstand $R_x$.
\subsection{Kapazitätsmessbrücke}
Nun wird eine Kapazitätsmessbrücke nach Abbildung\ref{fig:capbruecke} aufgebaut,
wobei auf den Widerstand $R_2$ verzichtet wird, da der Innenwiderstand des
unbekannten Kondensators klein genug ist.
Eine unbekannte Kapazität und ihr Widerstand wird bestimmt. Dafür werden die
Widerstände $R_2R_3$ und $R_4$ so variiert, dass die Brückenspannung minimiert
wird. Dies wird mit weitern $C_2$ zur Fehlerbestimmung wiederholt. Nun werden
ein weiterer Kondensator und eine RC Kombination ausgemessen. Für die Ausmessung
von der RC Kombination muss der Widerstand $R_2$ wieder eingebaut werden.
\subsection{Induktivitätsmessbrücke}
Mit einer Induktivitätsbrücke nach Abblidung\ref{fig:indbruecke} wird mit der
selben Methode eine unbekannte Induktivität bestimmt, nur das nicht $C_2$
ausgetauscht wird sondern $L_2$.
\subsection{Maxwell-Brücke}
Die selbe Induktivität wird eine Maxwell-Brücke nach Abbildung\ref{fig:Maxwellb}
aufgebaut und mit dieser vermessen. Dafür werden die Widerstände $R_3$ und $R_4$
variiert, bis die Brückenspannung minimal ist.
\subsection{Wien-Robinson-Brücke}
Eine Wien-Robinson-Brücke wird nach Abblidung\ref{fig:WBBruecke} aufgebaut. Die
Spannungen $U_{Br}(\omega)$ und $U_{S}(\omega)$ der Brückenschaltung werden für
mehr als $20$ Frequenzen im Bereich von $20-30000$ Hz gemessen und das Ergebnis
mit der Theorie verglichen. Danach wird der Klirrfaktor des Sinusgenerators bestimmt.
