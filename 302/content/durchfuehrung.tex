\section{Durchführung}
\label{sec:Durchführung}
Es wird eine Wheatonsche Brückenschaltung nach Abbildung\ref{fig:wheatonsche}
aufgebaut. Die Widerstände $R_3$ und $R_4$ werden so variiert, sodass das
Oszillioskop, das als Nullindikator dient, eine Minimale Spannung anzeigt.
Mit anderen Wiederständen $R_2$ wird die Messung zur Fehlerbestimmung wiederholt,
sowie mit einem anderen Widerstand $R_x$.
Nun wird eine Kapazitätsmessbrücke nach Abbildung\ref{fig:capbruecke} aufgebaut.
und eine unbekannte Kapazität und ihr Widerstand bestimmt. Dafür werden die
Widerstände $R_2R_3$ und $R_4$ so variiert, dass die Brückenspannung minimiert
wird. Dies wird mit weitern $C_2$ zur Fehlerbestimmung wiederholt. Mit einer
Induktivitätsbrücke nach Abblidung\ref{fig:indbruecke} wird mit der selben Methode
Eine unbekannte Induktivität Bestimmt. Danach wird die selbe Induktivität mit einer
Maxwell-Brücke nach Abbildung\ref{fig:Maxwellb} vermessen. Eine Wien-Robinson-Brücke
wird nach Abblidung\ref{fig:WBBruecke} aufgebaut. Die Frequenz der Brückenschaltung
wird im Bereich von $20-30000$ Hz gemessen und das Ergebnis mit der Theorie
verglichen. Dafür werden die Spannungen $U_{Br}(\omega)$ und $U_{S}(\omega)$
gemessen. Der 
