\section{Theorie}
\label{sec:Theorie}

\cite{sample}
Brückenschaltungen werden in der messtechnik eingesetzt da mit ihnen die Auflösung
der Messung erhöht werden kann.

In Abbildung\ref{} ist das Bild einer prinzipiellen Brückenschaltung zu sehen.
Die Potentialdifferenz $U$ wird zwischen den Punkten A und B gemessen.Aus der
Knotenregel ergibt sich das die Summe der Ströme an einer verzweigung $0$ ist.
\begin{equation}
  \sum \limits_{k} I_k= 0
  \label{eq:knotenregel}
\end{equation}
Aus der Maschenregel ergibt sich, dass die Summe aller Spannungen um eine Masche
$0$ ist. Daher gilt dies auch für die Produkte der Ströme und Widerstände nach
der Gleichung $U=RI$.
\begin{equation}
  \sum \limits_{k} U_k=  \sum \limits_{k} I_k R_k
  \label{eq:Maschenregel}
\end{equation}
Wird die Knotenregel\eqref{eq:knotenregel} für C und D angewendet gilt
\begin{equation}
  I_1=I_2
\end{equation}
\begin{equation}
  I_1=I_2    .
\end{equation}
Aus den beiden Maschen in der Abbildung\ref{} die, die Punkte A, B, C und D enthalten,
lässt sich mit der Maschenregel\eqref{eq:Maschenregel}
\begin{equation}
  U=-R_1 I_1 + R_3 I_3   .
\end{equation}
\begin{equation}
  -U=-R_2 I_2 + R_4 I_4   .
\end{equation}
\begin{equation}
  -U=-R_2 I_1 + R_4 I_3   .
\end{equation}
\begin{equation}
  U=\frac{R_2 R_3 - R_1 I_4}{R_3 + R_4}I_1   .
\end{equation}

\begin{equation}
  U=I_1 (R_1 + R_2)   .
\end{equation}
\begin{equation}
  U=\frac{R_2 R_3 - R_1 R_4}{R_3 + R_4)(R_4 + R_2) }U_S   .
\end{equation}
\begin{equation}
  R_1 R_4 =R_2 R_3   .
\end{equation}
\begin{equation}
\hat{Z}=X + iY
\end{equation}
\begin{align}
&\hat{Z}_C = \frac{-i}{\omega}  &  \hat{Z}_L=i\omega L & & \hat{Z}_R = R &
\end{align}
\begin{equation}
\hat{Z}_1\hat{Z}_4=\hat{Z}_2\hat{Z}_3
\end{equation}
\begin{equation}
X_1X_4-Y_1Y_4=X_2X_3-Y_2Y_3
\end{equation}
\begin{equation}
X_1Y_4+X_4Y_1=X_2Y_3+X_3Y_2
\end{equation}
\begin{equation}
R_x=R_2 \frac{R_3}{R_4}
\end{equation}
\begin{equation}
\hat{Z}_{C_{real}}=R-\frac{i}{\omega C}
\end{equation}
\begin{align}
&Y_1=\frac{-1}{\omega C_x} & Y_2=\frac{-1}{\omega C_2} & & Y_3=Y_4=0 &
\end{align}
\begin{equation}
R_x=R_2\frac{R_3}{R_4}
\end{equation}
\begin{equation}
C_x=C_2\frac{R_4}{R_3}
\end{equation}
\begin{equation}
\hat{Z}_{L_{real}}=R+i\omega L
\end{equation}
\begin{align}
&Y_1=\omega L_x  & Y_2=\omega L_2 & & Y_3=Y_4=0 &
\end{align}
\begin{equation}
R_x=R_2\frac{R_3}{R_4}
\end{equation}
\begin{equation}
L_x=L_2 \frac{R_3}{R_4}
\end{equation}
\begin{equation}
\hat{Z}_1=R_x+i \omega L_x
\end{equation}
\begin{equation}
\frac{}{\hat{Z}_4}=\frac{1}{R_4}+i\omega C_4 \text{oder}
\hat{Z}_4=\frac{R_4-i\omega C_4 R^2_4}{1+\omega^2 C^2_4 R^2_4}
\end{equation}
\begin{equation}
R_xR_4+\omega^2 R^2_4 C_4 L_x = R_2 R_3 \left(1+\omega^2 C^2_4 R^2_4 \right)
\end{equation}
\begin{equation}
-\omega R_x R^2_4 C_4 + \omega R_4 L_x=0
\end{equation}
\begin{equation}
R_xR_4\left(1+\omega^2 C^2_4 R^2_4 \right) =R_2 R_3 \left(1+\omega^2 C^2_4 R^2_4 \right)
\end{equation}
\begin{equation}
R_x= \frac{R_2 R_3}{R_4}
\end{equation}
\begin{equation}
L_x=R_2 R_3 C_4
\end{equation}
\begin{align}
\hat{Z}_1=2R' & \hat{Z}_2=R' &
\hat{Z}_3=R + \frac{1}{i \omega C}=\frac{i\omega C R + 1}{i \omega C}
\end{align}
\begin{equation}
\hat{U}_ {Br}=\frac{\hat{Z}_1\hat{Z}_4-\hat{Z}_2\hat{Z}_3}
{\left\{\hat{Z}_1+\hat{Z}_2 \right\}\left\{\hat{Z}_3 + \hat{Z}_4 \right\}}U_S
=\frac{\omega^2R^2C^2-1}{3(1-\omega^2 R^2 C^2+3i\omega RC)}U_S
\end{equation}
\begin{equation}
\frac{U_S}{U_{Br}}=\frac{1}{\omega}^2R^2C^2-1\left\{\,3(1-\omega^2 R^2 C^2)
+9i\omega RC \right\}
\end{equation}
\begin{equation}
\left|\frac{U_{Br}}{U_S}\right|^2=\frac{(\omega^2R^2C^2-1)^2}
{9\left\{(1-\omega^2 R^2 C^2)^2+9\omega^2R^2C^2\right\}}
\end{equation}
\begin{equation}
\omega_0=\frac{1}{RC}
\end{equation}
\begin{equation}
\Omega:=\frac{\omega}{\omega_0}
\end{equation}
\begin{equation}
\left|\frac{U_{Br}}{U_S}\right|^2=\frac{1}{9}\frac{(\Omega^2-1)^2}
{(1-\Omega^2)^2+9\Omega^2}
\end{equation}
\begin{equation}
\frac{\hat{U}_1-\hat{U}_S}{R}+\frac{U_1-U_{Br}}{R}=2i\omega C \hat{U}_1
\end{equation}
\begin{equation}
I_3+I_4=I_R
\end{equation}
\begin{equation}
(\hat{U}_S-\hat{U}_2)i\omega C+(\hat{U}_{Br}-\hat{U}_2)i\omega C
= \frac{2}{R}\hat{U}_2
\end{equation}
\begin{equation}
I_2=I_4
\end{equation}
\begin{equation}
(\hat{U}_2-\hat{U}_{Br})i\omega C+\frac{\hat{U}_1-\hat{U}_{Br}}{R}=0
\end{equation}
\begin{equation}
\mathfrak{U}_{Br}=U_{S}\frac{1-\omega^2 R^2 C^2 }
{1-\omega^2 R^2 C^2 + 4i\omega R C}
\end{equation}
\begin{equation}
\omega_0=\frac{1}{RC}
\end{equation}
\begin{equation}
\Omega := \frac{\omega}{\omega_0}
\end{equation}
\begin{equation}
\frac{U_{Br}}{U_S}=\frac{1-\Omega^2}{1-\Omega^2 +4i\Omega}
\end{equation}
\begin{equation}
\frac{U_{Br}}{U_S}=\frac{(1-\Omega^2)(1-\Omega^2-4i\Omega)}{(1-\Omega^2)^2+16\Omega^2}
\end{equation}
\begin{equation}
\left|\frac{U_{Br}}{U_S}\right|^2=\frac{(\Omega^2-1)^2}{(1-\Omega^2)^2+16\Omega^2}
\end{equation}
