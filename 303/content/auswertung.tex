\section{Auswertung}
\label{sec:Auswertung}
\subsection{Signale ohne Rauschen}
\label{sec:Signale ohne Rauschen}
Der rechte Ausgang des Referenc/Oscillator gibt eine konstante Spannung
von $2.4\si{\volt}$ aus. Der linke Ausgang liefert eine variable Spannung, die
als Rechteck oder sinusförmige Spannung abgegeben werden kann. Für verschiedene
Phasenverschiebung sehen die Signale wie in der folgenden
Abbildung\,\ref{fig:Signale ohne Rauschen} aus, wobei die Offsetphasenverschiebung
 $190°$ beträgt.
\begin{figure}
\centering
\begin{subfigure}{0.48\textwidth}
\centering
\includegraphics[height=4cm]{Bilder/or/or10.JPG}
\caption{Phasenwinkel $10°$.}
\label{fig:orp10}
\end{subfigure}
\begin{subfigure}{0.48\textwidth}
\centering
\includegraphics[height=4cm]{Bilder/or/or100.JPG}
\caption{Phasenwinkel $100°$.}
\label{fig:orp100}
\end{subfigure}
\begin{subfigure}{0.48\textwidth}
\centering
\includegraphics[height=4cm]{Bilder/or/or150.JPG}
\caption{Phasenwinkel $150°$.}
\label{fig:orp150}
\end{subfigure}
\begin{subfigure}{0.48\textwidth}
\centering
\includegraphics[height=4cm]{Bilder/or/or190.JPG}
\caption{Phasenwinkel $190°$.}
\label{fig:orp190}
\end{subfigure}
\begin{subfigure}{0.48\textwidth}
\centering
\includegraphics[height=4cm]{Bilder/or/or280.JPG}
\caption{Phasenwinkel $280°$.}
\label{fig:orp280}
\end{subfigure}
\caption{Signale ohne Rauschen.}
\label{fig:Signale ohne Rauschen}
\end{figure}
Die Abbildung\,\ref{fig:orp190} zeigt das Ergebnis nach dem Mischen, wenn $U_\text{sig}$
und $U_\text{ref}$ in Phase sind. Wie in der Theorie ist hier der Betrag einer Sinus-Schwingung
zu sehen. Wenn das Ausgangssignal über den Tiefpass integriert wird, sieht das
Signal aus wie in der Abblidung\,\ref{fig:int}.
\begin{figure}
  \centering
  \includegraphics[height=5cm]{Bilder/int/INT.JPG}
  \caption{Phasenwinkel $190°$.}
  \label{fig:int}
\end{figure}
\subsection{Messreihe ohne Rauschen}
\label{sec:Messreihe ohne Rauschen}
Die Messwerte aus der ersten Messreihe ohne Signalstörungen, sind im folgenden
Diagramm eingetragen sowie die Theoriekurve, die um den Offset von $12\si{\volt}$
verschoben wurde.
\begin{align*}
U_{out}=\frac{2}{\pi}U_0cos(\phi)
\end{align*}
 Hierbei ist $U_0=6.2\si{\volt}$. Das Ergebnis der Messreihe ist
mit der Theoriekurve gut vergleichbar, wenn der Offset nicht beachtet wird.
Daher lässt sich die Funktion des Lockinverstärkers verifizieren.
\begin{figure}
  \centering
  \includegraphics[height=8cm]{or_signal.pdf}
  \caption{Messreihe ohne Rauschen.}
  \label{fig:Mor}
\end{figure}

\subsection{Signale mit Rauschen}
\label{sec:Signale mit Rauschen}
Die folgende Abbildung\,\ref{fig:Signale mit Rauschen} zeigt stark verrauschte
Ausgangssignale des Mischers. Es lässt nur noch grob mit den unverrauschten
Signalen vergleichen. Der Offsetphasenwinkel hat sich um $180°$ verschoben und beträgt jetzt $10°$.
Die Abbildung\,\ref{fig:rp10} zeigt das Ausgangssignal bei dem $U_\text{sig}$ und
$U_\text{ref}$ in Phase sind.

\begin{figure}
\centering
\begin{subfigure}{0.48\textwidth}
\centering
\includegraphics[height=4cm]{Bilder/r/r10.JPG}
\caption{Phasenwinkel $10°$.}
\label{fig:rp10}
\end{subfigure}
\begin{subfigure}{0.48\textwidth}
\centering
\includegraphics[height=4cm]{Bilder/r/r100.JPG}
\caption{Phasenwinkel $100°$.}
\label{fig:rp100}
\end{subfigure}
\begin{subfigure}{0.48\textwidth}
\centering
\includegraphics[height=4cm]{Bilder/r/r150.JPG}
\caption{Phasenwinkel $150°$.}
\label{fig:rp150}
\end{subfigure}
\begin{subfigure}{0.48\textwidth}
\centering
\includegraphics[height=4cm]{Bilder/r/r190.JPG}
\caption{Phasenwinkel $190°$.}
\label{fig:rp190}
\end{subfigure}
\begin{subfigure}{0.48\textwidth}
\centering
\includegraphics[height=4cm]{Bilder/r/r280.JPG}
\caption{Phasenwinkel $280°$.}
\label{fig:rp280}
\end{subfigure}
\caption{Signale mit Rauschen.}
\label{fig:Signale mit Rauschen}
\end{figure}

\subsection{Messreihe mit Rauschen}
\label{sec:Messreihe mit Rauschen}
Bei der Messreihe mit verrauschten Signalen, gab es keinen messbaren Offset. Die
Messwerte stimmen mit der Theoriekurve überein bis auf eine kleine Verschiebung
im Phasenwinkel. Daher lasst sich bestätigen das der Lockinverstärker fähig ist,
stark verrauschte Signale zu filtern.
\begin{figure}
  \centering
  \includegraphics[height=8cm]{r_signal.pdf}
  \caption{Messreihe mit Rauschen.}
  \label{fig:Mr}
\end{figure}

\subsection{Messreihe zur LED Intensität}
\label{sec:Messreihe zur LED intensität}
Bei der Messreihe zur LED Intensität in Abhängigkeit zum Abstand lässt sich das
Licht bis zu einer Entfernung von $1.66\,\si{\meter}$ nachweisen. Im Diagramm
\ref{fig:LED} lässt sich erkennen, dass die Messdaten wie die Theoriekurve mit
$ 4\pi r^2$ abfallen. Da der Sensor auch das Licht der Raumbeleuchtung auf nimmt,
 lässt sich mit diesem Ergebnis die Funktion stark verrauschte Signale zu
Filtern bestätigen.

\begin{figure}
  \centering
  \includegraphics[height=8cm]{LED.pdf}
  \caption{Messreihe abstands.}
  \label{fig:LED}
\end{figure}
