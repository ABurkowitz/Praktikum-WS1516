\section{Auswertung}
\label{sec:Auswertung}
Der rechte Ausgang des Referenc/Oscillator gibt es eine konstante Spannung
von $2.4\si{\volt}$ der linke Ausgang liefert eine variable Spannung die
als Rechteck oder sinusförmige Spannung abgegeben werden kann. Für verschiedene
Phasenverschiebung sehen die Signale wie in den folgenden
Bildern aus.
\begin{figure}
\centering
\begin{subfigure}{0.48\textwidth}
\centering
\includegraphics[height=5cm]{Bilder/or/or10.JPG}
\caption{Phasenwinkel $10°$.}
\label{fig:orp10}
\end{subfigure}
\begin{subfigure}{0.48\textwidth}
\centering
\includegraphics[height=5cm]{Bilder/or/or100.JPG}
\caption{Phasenwinkel $100°$Das TU-Logo.}
\label{fig:orp100}
\end{subfigure}
\begin{subfigure}{0.48\textwidth}
\centering
\includegraphics[height=5cm]{Bilder/or/or150.JPG}
\caption{Phasenwinkel $150°$.}
\label{fig:orp150}
\end{subfigure}
\begin{subfigure}{0.48\textwidth}
\centering
\includegraphics[height=5cm]{Bilder/or/or190.JPG}
\caption{Phasenwinkel $190°$.}
\label{fig:orp190}
\end{subfigure}
\begin{subfigure}{0.48\textwidth}
\centering
\includegraphics[height=5cm]{Bilder/or/or280.JPG}
\caption{Phasenwinkel $280°$.}
\label{fig:orp280}
\end{subfigure}
\end{figure}
Wenn das Ausgangssignal über den Tiefpass für $\Phi=$ integriert wird sieht das
Signal aus wie in dem folgenden Bild.
\begin{figure}
  \centering
  \includegraphics[height=5cm]{Bilder/int/INT.JPG}
  \caption{Phasenwinkel $280°$.}
  \label{fig:int}
\end{figure}
Die Messwerte aus der ersten Messreihe ohne Signalstörungen, sind im folgenden
Diagramm eingetragen sowie die Theoriekurve
\begin{align*}
U_{out}=\frac{2}{\pi}U_0cos(\phi)
\end{align*}
,die um den Offset von $12\si{\volt}$ verschoben wurde.
\begin{figure}
  \centering
  \includegraphics[height=8cm]{or_signal.pdf}
  \caption{Messreihe ohne Rauschen.}
  \label{fig:Mor}
\end{figure}






\begin{figure}
\centering
\begin{subfigure}{0.48\textwidth}
\centering
\includegraphics[height=5cm]{Bilder/r/r10.JPG}
\caption{Phasenwinkel $10°$.}
\label{fig:rp10}
\end{subfigure}
\begin{subfigure}{0.48\textwidth}
\centering
\includegraphics[height=5cm]{Bilder/r/r100.JPG}
\caption{Phasenwinkel $100°$.}
\label{fig:rp100}
\end{subfigure}
\begin{subfigure}{0.48\textwidth}
\centering
\includegraphics[height=5cm]{Bilder/r/r150.JPG}
\caption{Phasenwinkel $150°$.}
\label{fig:rp150}
\end{subfigure}
\begin{subfigure}{0.48\textwidth}
\centering
\includegraphics[height=5cm]{Bilder/r/r190.JPG}
\caption{Phasenwinkel $190°$.}
\label{fig:rp190}
\end{subfigure}
\begin{subfigure}{0.48\textwidth}
\centering
\includegraphics[height=5cm]{Bilder/r/r280.JPG}
\caption{Phasenwinkel $280°$.}
\label{fig:rp280}
\end{subfigure}
\end{figure}
\begin{figure}
  \centering
  \includegraphics[height=8cm]{r_signal.pdf}
  \caption{Messreihe mit Rauschen.}
  \label{fig:Mr}
\end{figure}


\begin{figure}
  \centering
  \includegraphics[height=8cm]{LED.pdf}
  \caption{Messreihe abstands.}
  \label{fig:LED}
\end{figure}
