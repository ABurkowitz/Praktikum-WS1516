\section{Anhang}
\label{Anhang}
\subsection{Dreieckspannung}
\label{Dreieckspannung}
Die Dreieckspannung wird parametriesiert durch die Funktion
\begin{equation*}
  f(t)=\left\{
  \begin{array}{l l}
    &\frac{4Kt}{T}+K   \;\;\;\;\;\;\frac{-T}{2}<t<0 \\
    &\frac{-4Kt}{T}+K  \;\;\;\;\;0<t<\frac{T}{2}
  \end{array}
  \right.
\end{equation*}
wird die Funktion in \eqref{} eingesetzt erhält man
\begin{equation*}
  a_n=\frac{2}{T}\int^{\frac{T}{2}}_0 \;t \cos(\frac{2n \pi t}{T} ) + K dt
  \int^0_{\frac{-T}{2}} \; t \cos(\frac{2n\pi t}{T} ) + K dt
\end{equation*}
daraus folgt
\begin{align*}
  a_n=\frac{4K}{T}\left(\left[ \frac{T}{2n\pi}\sin(\frac{2nt\pi}{T})\right]^\frac{T}{2}_0
  -\frac{T}{4n^2\pi^2}\left[-\cos(\frac{2nt\pi}{T})\right]^0_\frac{-T}{2} \right)
\end{align*}
da dies eine gerade Funktion ist, entfallen die Koeffizienten der Sinus-Terme.
so das für die Koeffizienten

\begin{equation}
  4K\left(\frac{1}{n^2\pi^2}(\cos(n\pi)-1)\right)
\end{equation}
gilt. Daher sind die Koeffizienten für gerade n
\begin{equation*}
a_n=\frac{-8K}{n^2}
\end{equation*}
und für ungerade n
\begin{equation*}
  a_n=0.
\end{equation*}
\subsection{Rechteckspannung}
\label{Rechteckspannung}
Die Rechteckspannung lasst sich durch die Funktion
\begin{equation*}
  f(t)=\left\{
  \begin{array}{l l}
    &K   \;\;\;\;\;\;\frac{-T}{2}<t<0 \\
    &-K  \;\;\;\;\;0<t<\frac{T}{2}
  \end{array}
  \right.
\end{equation*}
beschreiben. bei dieser Funktion handelt es sich um eine ungerade Funktion, daher
sind die Koeffizienten der Kosinuns-Funktion $0$. Für die Koeffizienten gilt
\begin{align*}
  b_n=&\frac{2}{T}\left(\int^{\frac{T}{2}}_0\;K\sin(\frac{2nt\pi}{T})dt+
\int^0_{\frac{-T}{2}}\;K\sin(\frac{2nt\pi}{T})dt\right)\\
  b_n=&2\left[ \frac{K}{n\pi}\cos(\frac{2nt\pi}{T})\right]^{\frac{T}{2}}_0\\
  b_n=&\frac{2K}{n\pi}(\cos(n\pi)+1)\;.
\end{align*}
Daher gilt für gerade n
\begin{equation}
  b_n=\frac{4K}{n\pi}
\end{equation}
 und für ungerade n
\begin{equation}
  b_n=0
\end{equation}
\subsection{Sägezahnspannung}
\label{Saegezahnspannung}
Die Sägezahnspannung ist durch folgende Funktion beschrieben.
\begin{equation*}
  f(t)=\left\{
  \begin{array}{l l}
    &\frac{Kt}{T}-K   \;\;\;\;\;\frac{-T}{2}<t<0 \\
    &\frac{Kt}{T}   \;\;\;\;\;\;\;\;\;\;\;\;\;0<t<\frac{T}{2}
  \end{array}
  \right.
\end{equation*}
Da es sich um eine ungerade Funktion handelt entfallen die Kosinus-Terme. Für die
anderen gelten
\begin{equation*}
  b_n=\frac{2}{T}\left(\int^{\frac{T}{2}}_0\;\left(\frac{Kt}{T}-K\right)
  \sin\left(\frac{2nt\pi}{T}\right)dt +\int^0_{\frac{-T}{2}}\frac{Kt}{T}
  \sin\left(\frac{2nt\pi}{T}\right)dt\right)
\end{equation*}
\begin{equation*}
  b_n=\left[\frac{\frac{8n\pi}{T^2}\sin(\frac{2nt\pi}{T})(-\frac{4Kt}{T}+K)
  -\cos(\frac{2nt\pi}{T}+T)}{\frac{4n\pi}{T^2}}\right]^{\frac{T}{2}}_0
\end{equation*}
\begin{equation*}
  b_n=\frac{-4K}{\pi^2n^2}(\cos(n\pi)-1)
\end{equation*}
so dass für die geraden n
\begin{equation}
  b_n=0
\end{equation}
gilt und für die ungeraden n
\begin{equation}
  b_n=\frac{-8K}{n^2\pi^2}
\end{equation}
