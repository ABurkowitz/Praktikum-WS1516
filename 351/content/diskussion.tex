\section{Diskussion}
\label{sec:Diskussion}
Die bei der Fourieranalyse abgelesenen Fourierkoeffizenten stimmen größtenteils
sehr gut mit den Theoriewerten überein. Die Abweichung der Messwerte von den
Theoriewerten wird tendentiell größer, je kleiner die Fourieramplituden werden,
was aber keinesfalls überraschend ist. Abgesehen von Messwerten der 17. Oberwelle
sind die relativen Abweichungen meist sehr klein, nämlich (meist deutlich) unter
10\%.

Bei der Fouriersynthese für die Rechteckschwingung und die Sägezahnschwingung
wurden die Amplituden mit dem Oszilloskop gemessen, bei der Dreiecksschwingung
jedoch mit einem Millivoltmeter, da die Messgenauigkeit des Oszilloskops nicht
mehr ausreichte. Dies führt dazu, dass die Amplituden für die Dreiecksschwingung
sehr viel genauer eingestellt werden konnten.\newline
Wie man den Grafiken der synthetisierten Funktionen \ref{fig:rechteck_synthese},
\ref{fig:saegezahn_synthese} und \ref{fig:dreieck_synthese} entnehmen kann, konnte
die Dreiecksspannung am besten angenähert werden. Dies bestätigt die Erwartung,
da die Fourierkoeffizienten dabei mit $\frac{1}{n^2}$ abfallen, sodass höhere
Fourierkoeffzienten viel weniger ins Gewicht fallen. Zudem hat die Dreicksspannung
keine Unstetigkeitsstellen, an denen das Gibbsche Phänomen auftreten kann.

Insgesamt bestätigt dieser Versuch das zu erwartende Verhalten bei einer
Fourieranalyse oder Fouriersynthese, obwohl nur die ersten neun Fourierkoeffizienten
jeweils betrachten oder eingestellt wurden anstatt unendlich viele Fourierkoeffzienten
zu berücksichtigen.
