\section{Durchführung}
\label{sec:Durchführung}
\subsection{Fourieranalyse}
Es werden die ersten 9 von null verschiedenen Fourieramplituen einer Rechteck-,
Sägezahn- und Dreiecksschwinung gemessen. Die Messwerte sind in Tabelle
\ref{fig:messwerte1} aufgeführt. Die gemessenen Werte werden mit den zuvor
berechneten Fourierkoeffizenten verglichen.


\subsection{Fouriersynthese}
Nun werden die Spannungsamplituden der einzelnen Oberwellen für drei verschiedene
Schwingungen (Rechteck-, Sägezahn- und Dreiecksschwingung) so eingestellt, dass
die Fouriersynthese die gewünschte Schwingungsform liefert. Dazu werden die ersten
neun oder zehn Oberwellen verwendet.
Die Summenschwingung wird mit einem Oszilloskop betrachtet. Die durch die
Fouriersynthese entstandene Schwingungen sind in den Abbildungen
\ref{fig:rechteck_synthese}, \ref{fig:saegezahn_synthese} und \ref{fig:dreieck_synthese}
abgebildet.
