\section{Theorie}
\label{sec:Theorie}
\subsection{Fouriersche Theorem}
Das Fourier Theorem besagt, dass die Reihe
\begin{equation}
\frac{1}{2}a_0+\sum_{n=0}^\infty\left(a_n \cos(n\frac{2\pi}{T}t)+b_n\sin(n\frac{2\pi}{T}t)\right)
\label{fuireihe}
\end{equation}
eine periodische Funktion beschreibt wenn die Reihe konvergiert und für die
Koeffizienten $a_n$ und $b_n$
\begin{equation}
  a_n=\frac{2}{T}\int_a^b\,\,f(t)\cos(n\frac{2\pi}{T}t)
  \label{fuikoeffizient_a}
\end{equation}
\begin{equation}
  b_n=\frac{2}{T}\int_a^b\,\,f(t)\sin(n\frac{2\pi}{T}t)
  \label{fuikoeffizient_b}
\end{equation}
gilt. In der Fourier Entwicklung treten die Frequenz $\nu=\frac{1}{T}$, sowie ihre
ihre ganzzahligen Vielfachen auf. Bei geraden Funktionen fallen alle $b_n$ weg
da gilt $f(t)=f(-t)$ und dies bei sinus Funktionen nicht gegeben ist. bei
ungeraden Funktionen fallen alle $a_n$ weg, da $f(t)=-f(-t)$ gilt und ein cosinus
diese Bedingung nicht erfüllt. $a_n$ und $b_n$ müssen nach Gleichung
\eqref{fuireihe} für $\lim\limits_{n \to \infty}$ gegen Null gehen, damit die
Reihe Konvergiert. Ein Beispiel dafür ist in Abbildung \ref{fig:Fouriertrafo}
dargestellt. Wenn $f(t)$ stetig ist folgt die Gleichmäßige Konvergenz. Ist $f(t)$
unstetig so lässt sie sich an dieser Stelle nicht mit der Fourier-Reihe
approximieren. Es tritt das Gibbsche Phänomen auf, das die Abweichung beschreibt.
\begin{figure}
  \centering
  \includegraphics[width=0.4\textwidth]{Fouriertrafo.png}
  \caption{Fouriertransformation \cite{sample}.}
  \label{fig:Fouriertrafo}
\end{figure}
\subsection{Fourier-Transformation}
\label{sec:Fourier-Transformation}
Mit der Fouruer-Transformation lässt sich das gesamte Frequenzspektrum einer
zeitabhängigen Funktion bestimmen. Die Fourier-Transformation ist definiert durch:
\begin{equation}
  g(\nu)= \int_{-\infty}^\infty f(t)\exp(i\nu t) \mathrm{d}t
\end{equation}
Sollte $f(t)$ periodisch sein, ist $g(\nu)$ eine konvergierende Reihe von
\delta-Funktionen wie die in Abbildung \ref{fig:Fouriertrafo}. Die Umkehrfunktion
\begin{equation}
  f(t)= \frac{1}{2\pi}\int_{-\infty}^\infty g(\nu)\exp(i\nu t) \mathrm{d}\nu
\end{equation}
ergibt aus dem Frequenzspektrum $g(\nu)$ die Funktion $f(t)$ wieder. In der
Praxis ist eine genaue Fourier-Transformation nicht möglich, da über einen unendlichen
Zeitraum integriert werden muss. Mit endlichen Integrationsgenzen ist keine
Periodizität mehr gegeben, dadurch ergibt sich aus der Transformation keine Reihe
von $\delta$-Funktionen sondern überall stetige Funktionen. Zudem entstehen
Nebenmaxima zu den ursprünglichen Hauptmaxima.
\subsection{Fourier-Analyse}
\label{sec:Fourier-Analyse}
Mit der Fourier-Analyse, werden Signale in Fourier-Koeffizienten aufgeteilt, mit
denen durch die Gleichungen \eqref{fuikoeffizient_b}, \eqref{fuikoeffizient_b}
und \eqref{fuireihe} die analysierten Schwingungen approximiert werden können.



\cite{sample}
