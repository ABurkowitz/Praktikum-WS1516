\section{Diskussion}
\label{sec:Diskussion}
Die Ausgleichsrechnung zu den Werten der gedämpften Schwingung aus Abbildung
\ref{fig:termodruck} wurde um den Fitparameter $E$ erweitert, da die Werte
der Spannung verschoben sind. Dies lässt sich durch eine Offsetspannung oder
eine falsche Einstellung im Oszilloskop erklären.
Die Berechneten Werte für $R_{eff}$ liegen, wenn man den Innenwiderstand des
Sinusgenerators mit beachtet, nahe bei den erwarteten Werten. Die
Abweichungen des Widerstandes für den apperiodischen Grenzfall entstehen
wahrscheinlich durch die unbeachteten und unbekannten Widerstände, wie z.B.
der Spule, und dadurch, dass die Genauigkeit zur Bestimmung des apperiodischen
Grenzfalls mit dem Oszilloskop begrenzt ist. Die große Abweichung von $\omega_0$
zum Theoriewert lässt sich nur durch einen Systematischen oder Rechenfehler erklären
da eine Abweichung von $84.6\%$ sich nicht nur durch Innenwiderstände erklären.
Die Abweichung von
$\omega_{+_e}-\omega_{-_e}$ zum Theoriewert lässt sich nur Teilweise durch die
begrenzte Anzahl an Messwerten erklären. Auch hier könnte der vorhergenannte
 systematischerfehler die Abweichung verursachen.
Auch ist, wie bei den anderen
Berechnungen, der Einfluss der unbekannten Schaltungswiderstände zu beachten.
