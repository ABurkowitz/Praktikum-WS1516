\section{Diskussion}
\label{sec:Diskussion}
Die Ausgleichsrechnung zu den Werten der gedämpften Schwingung aus Abbildung
\ref{fig:termodruck} wurde um den Fitparameter $E$ erweitert, da die Werte
der Spannung verschoben sind. Dies lässt sich durch eine Offsetspannung oder
eine falsche Einstellung im Oszilloskop erklären.
Die berechneten Werte für $R_{\symup{eff}}$ liegen, wenn der Innenwiderstand des
Sinusgenerators mit beachtet wird, nahe bei den erwarteten Werten. Die
Abweichungen des Widerstandes für den apperiodischen Grenzfall entstehen
 durch die unbeachteten und unbekannten Widerstände, wie z.B.
der Spule, und dadurch, dass die Genauigkeit zur Bestimmung des apperiodischen
Grenzfalls mit dem Oszilloskop begrenzt ist.
Auch bei den anderen Berechnungen, ist der Einfluss der unbekannten
Schaltungswiderstände zu beachten. Es gibt bei den berechneten Werten
eine maximale Abweichung von $38.5\%$ bei der Breite der Resonanzkurve.
Die in Abbildung \ref{fig:Kondensatorspannung} dargestellten Werte,
stimmen mit dem erwarteten Verlauf überein.
Die meisten in Abbildung \ref{fig:phasenverschiebung} abgebildeten Werte stimmen mit dem
erwarteten Verlauf eines Arkustangens überein. Nur die letzten drei Messwerte
weichen ab.
