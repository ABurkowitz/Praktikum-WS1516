\section{Auswertung}
\label{sec:Auswertung}

Für die Berechnungen und Experimente wurden die Eigenschaften und Einstellungen
\begin{align}
  L=&(23.954\pm0.001)\,\si{\milli\henry}  \\
  C=&(0.7932\pm0.0001)\,\si{\nano\farad}  \\
  C_{sp}=&(0.028\pm0.001)\,\si{\nano\farad}\\
  R=&(48\pm1)\,\si{\ohm}\\
  \nu=&(32.05\pm0.01)\,\si{\kilo\hertz}\\
\end{align}
benutzt. Hier ist $C_{sp}$ der capazitäre Anteil der der Spule $L$ und $\nu_{res}$ die eingestellte
Frequenz.
In der Tabelle \ref{tab:5a} sind für verschiedene Koppelkondensatoren die Verhältnisse
zwischen Schwebungs- und Schwingfrequenz, die durch die Anzahl der Maxima
in einer Schwebungsperiode gegeben ist, angegeben. Die Theoriewerte sind durch das
Verhältnis
\begin{equation*}
  n_{max}=\frac{\nu^{+}+\nu^{-}}{2(\nu^{-}-\nu^{+})}
\end{equation*}
gegeben. Die Anzahl der Schwingungsmaxima in einer Periode, waren bei den
Koppelkondensatoren mit $2.2 \si{\nano\farad}$ und $1.0 \si{\nano\farad}$
 nicht ablesbar, weshalb diese nicht aufgeführt sind.
\begin{table}
  \centering
  \begin{tabular}{c c c c}
    \toprule
    Koppelkondensator $C_k$ in \si{\nano\farad} & Maxima einer Schwingung $n_{max}$& Theoriewert
    & Abweichung \\
    \midrule
    12.0 & 10\pm1  & 15.6\pm2.9 & 35.8\%\\
    10.0 & 8\pm1   & 13.2\pm2.4 & 39.4\%\\
    8.2  & 7\pm1   & 11.0\pm2.0 & 36.4\%\\
    6.8  & 6\pm1   &  9.3\pm1.7 & 30.7\%\\
    4.7  & 4\pm1   &  6.7\pm1.2 & 40.3\%\\
    2.7  & 2\pm1   &  4.2\pm0.7 & 52.3\%\\
    \bottomrule
  \end{tabular}
  \caption{Ergebnisse zur Messung von Schwing- und Schwebefrequenz Verhältnis}
  \label{tab:5a}
\end{table}
Die Tabelle \ref{tab:5b} gibt die errechneten und experimentell bestimmten
Fundamentalfrequenzen für verschiedene Koppelkondensatoren an. Berechnet wurden
sie mit den Formeln \eqref{eqn:frequenz_puls} für $\nu^{+}$
und \eqref{eqn:frequenz_minus} für $\nu^{-}$. Die experimentell bestimmten Werte
stammen aus dem Versuchsaufbau nach Abbildung \ref{fig:messung1}.
\begin{table}
  \centering
  \begin{tabular}{c c c c c c c}
    \toprule
    $C_k$ in \si{\nano\farad} & $\nu^{+}_E$ in \si{\hertz} & $\nu^{+}_T$ in \si{\hertz}
    & Abweichung &$\nu^{-}_E$ in \si{\hertz} & $\nu^{-}_T$ in \si{\hertz} & Abweichung \\
    \midrule
    12.0 & 32050\pm10 & 35884\pm2 & 10.7\% & 39680\pm10 & 38300\pm50  & 3.60\%  \\
    10.0 & 32050\pm10 & 35884\pm2 & 10.7\% & 40320\pm10 & 38700\pm50  & 4.19\%  \\
    8.2  & 32050\pm10 & 35884\pm2 & 10.7\% & 41670\pm10 & 39300\pm70  & 6.03\%  \\
    6.8  & 32050\pm10 & 35884\pm2 & 10.7\% & 43860\pm10 & 40000\pm80  & 9.00\%  \\
    4.7  & 32050\pm10 & 35884\pm2 & 10.7\% & 47170\pm10 & 41700\pm110 & 13.12\% \\
    2.7  & 32050\pm10 & 35884\pm2 & 10.7\% & 56180\pm10 & 45500\pm170 & 23.47\% \\
    2.2  & 32050\pm10 & 35884\pm2 & 10.7\% & 60240\pm10 & 47400\pm200 & 27.09\% \\
    1.0  & 32050\pm10 & 35884\pm2 & 10.7\% & 79620\pm10 & 58000\pm400 & 37.28\% \\
    \bottomrule
  \end{tabular}
  \caption{Messergebnisse und berechnete Werte, für die Messung der Fundamentalschwingungen
  nach Abbildung \ref{fig:messung1} .}
  \label{tab:5b}
\end{table}
In der Tabelle \ref{tab:5c} werden die experimentell bestimmten Fundamentalfrequenzen
aus der dritten Messung angegeben sowie die relative Abweichung. Die Fundamentalfrequenzen
berechnen sich hierbei durch
\begin{align*}
  f_E^-=f_s+\delta t_1(f_E-f_s)
  f_E^-=f_s+\delta t_2(f_E-f_s)
\end{align*}
\begin{table}
  \centering
  \begin{tabular}{c c c c c c c}
    \toprule
    Koppelkondensator & $\delta t_1$
    & $\nu^{+}_E$ \si{\hertz} & Abweichung & $\delta t_2$
    & $\nu^{-}_E$ \si{\hertz} & Abweichung\\
    $C_k$ in \si{\nano\farad} & in \si{\milli\second} & & & in \si{\milli\second}
     & & \\
    \midrule
    12.0 & 296\pm1 & 33490\pm12 & 6.7\% & 720\pm1 & 39510\pm12 & 3.2\%   \\
    10.0 & 304\pm1 & 33490\pm12 & 6.7\% & 540\pm1 & 40850\pm12 & 5.6\%   \\
    8.2  & 300\pm1 & 33260\pm12 & 7.3\% & 508\pm1 & 42640\pm12 & 8.5\%   \\
    6.8  & 300\pm1 & 33040\pm12 & 7.9\% & 432\pm1 & 43980\pm12 & 10.0\%  \\
    4.7  & 296\pm1 & 33490\pm12 & 6.7\% & 394\pm1 & 48220\pm12 & 15.6\%  \\
    2.7  & 298\pm1 & 33490\pm12 & 6.7\% & 382\pm1 & 56700\pm12 & 25.6\%  \\
    2.2  & 300\pm1 & 33930\pm12 & 5.4\% & 366\pm1 & 60270\pm12 & 27.2\%  \\
    1.0  & 300\pm1 & 33040\pm12 & 7.9\% & 354\pm1 & 80350\pm13 & 38.5\%  \\
    \bottomrule
  \end{tabular}
  \caption{Fundamentalfrequenzen des Schwingkeises}
  \label{tab:5c}
\end{table}
