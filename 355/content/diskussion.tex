\section{Diskussion}
\label{sec:Diskussion}
Bei der Bestimmung der Verhältnisse der Schwebungs- und Schwingfrequenz ist ein
Ablesen der Maxima nur schwer oder garnicht möglich gewesen, weshalb keine Angaben
für die Koppelkondensatoren mit $2.2\si{\nano\farad}$ und $1.0\si{\nano\farad}$
gemacht wurden.
Bei der Bestimmung der Eigenschwingfrequenzen scheint ein systematischer Fehler
vorzuliegen, da der Theoriewert von $\nu^{+}$ immer größer ist, als die experimentell
bestimmten Werte. Bei $\nu^{-}$ ist der Theoriewert immer kleiner. Dies gilt für
beide Messmethoden. Das könnte durch unbekannte kapazitäre Anteile wie z.B. $C_{sp}$
der Schaltung veruracht werden. Bei den Frequenzen und Zeiten werden die Letzten
ablesbaren Stellen mit einem Fehler von $1$ belegt. Damit werden die die Werte
berechnet.
Die Werte für $\nu_E^+$ bleiben wie erwart für alle Kondensatoren gleich.
Die erste Methode scheint die genauere zu sein, da hier die relativen
Fehler leicht geringer sind.
